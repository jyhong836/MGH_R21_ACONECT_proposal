\documentclass[11pt]{article}

% Packages
\usepackage[utf8]{inputenc}
\usepackage{helvet}  % Arial font
\usepackage[margin=0.5in]{geometry}  % 0.5 inch margins on all sides
\usepackage[hidelinks]{hyperref}
\usepackage{xcolor}
\usepackage{enumitem}
\usepackage{setspace}
\usepackage[numbers,sort&compress,super]{natbib}  % For citations
\usepackage{soul}  % For underlining with \ul
\usepackage{amssymb}  % For mathematical symbols including \geq and \leq
\usepackage{titlesec}  % For customizing section formatting

% Set font to Arial (Helvetica)
\renewcommand{\familydefault}{\sfdefault}

% Change section numbering to A, B, C, etc. and make titles uppercase
\renewcommand{\thesection}{\Alph{section}.}
\titleformat{\section}{\normalfont\bfseries}{\thesection}{1em}{\MakeUppercase}

% Customize title formatting
\makeatletter
\renewcommand{\@maketitle}{%
  \newpage
  \null
  \vfill
  \begin{center}%
    {\fontsize{20pt}{24pt}\selectfont\bfseries \@title \par}%
  \end{center}%
  \vfill
}
\makeatother

% Reduce spacing after title
\titlespacing*{\section}{0pt}{0.5em}{0.3em}

% Remove paragraph indentation
\setlength{\parindent}{0pt}

% Set line spacing
\setstretch{1.15}

% Title and document setup
\title{AI-CONECT: Developing Conversational AI for Early Dementia Prevention in Socially-Isolated Older Adults}
\author{}
\date{}

\begin{document}

\maketitle

\clearpage

\textbf{Specific Aims}

\textbf{Significance}. Alzheimer's disease and related dementias (ADRD)
remain a major and growing public health burden, and the disease's
multifactorial biology continues to outpace the impact of available
therapies \cite{zhang2024recent}. Current U.S. Food and Drug Administration-approved
symptomatic treatments offer at best modest and often short-lived
benefits, while newer disease-modifying anti-amyloid immunotherapies are
limited to biomarker- confirmed early Alzheimer's disease and are
accompanied by restricted patient eligibility and significant treatment
burden with ongoing cost-effectiveness concerns \cite{mangalagiu2025pharmacological}. Thus, it is
important to develop accessible and affordable prevention strategies
that address modifiable dementia risk factors early in the disease
course.

\textbf{Critical Barriers.} Social isolation is a recognized, modifiable
risk factor for cognitive decline and dementia in older adults
\cite{evans2019social,kallianpur2023weak,nasem2020social,penninkilampi2018association,poey2017social,shen2022associations}, yet effective interventions that can deliver sustained,
high-fidelity social engagement at the population scale remain limited.
The I-CONECT clinical trial (NCT02871921) provides evidence that
semi-structured conversational engagement with cognitive stimulation is
feasible and can improve social connection--related outcomes among
socially isolated older adults with Mild Cognitive Impairment (MCI)
\cite{dodge2024internet}. However, interviewer-delivered programs are constrained by
workforce availability and high per-participant cost, rendering a
persistent barrier to broad dissemination.

\textbf{Solutions.} To enhance the accessibility and affordability of
the I-CONECT intervention, we propose to develop an AI-led
conversational intervention (AI-CONECT) by leveraging advanced
Generative AI to simulate the human-delivered I-CONECT conversational
intervention. Similar to the I-CONECT intervention, the AI chatbot will
provide the same conversation structure, cognitive stimulation with
images and pre-selected topics, and conversation strategies to engage
participants and stimulate their cognitive functions. The multi-modality
interaction of the chatbot includes on-demand voice input/output and
visual emotion responses, which minimizes the gap to human interactions.
Our preliminary result \cite{hong2024aconect} has shown that Large Language Models
(LLMs) with customized prompts can increase participants' engagement in
simulated conversations. However, the system in \cite{hong2024aconect} is
under-optimized for engaging older adults: limited modality or device
adaptation for a technique-unfamiliar population. Moreover, the
feasibility of the AI-led intervention has not been evaluated among the
targeted population yet. Thus, our specific aims in this proposal are:

\textbf{Aim 1: Develop a user-friendly multi-modality social chatbot
system (AI-CONECT) for executing the I-CONECT intervention.} We will
develop the chatbot for two design goals. (i) \textbf{Intervention
compliance}: We will use an agentic design to enable the AI chatbot to
follow the I-CONECT dialogue workflow, present chat materials, and
generate engaging responses adaptively based on the users' responses
\cite{hong2024aconect}. (ii) \textbf{User friendliness}: New tech has great barriers
for older adults, particularly those with MCI, to use. To reduce the
barrier, we will integrate on-demand voice input/output with visual
emotion-responsive interaction. We will also deploy a ready-to-use,
older-adult-friendly tablet paired with an emotion-aware social robot
that provides real-time affective feedback to support sustained
engagement \cite{zhao2025transferring}. (iii) \textbf{Safety}: Older adults with MCI and
living alone have limited awareness of AI risks and could have higher
risks in use. We will create real-time guardrails on the conversation to
monitor critical situations like low emotion, suicide, misinformation,
or financial risks \cite{xiang2024guardagent}.

\textbf{Aim 2: Conduct a feasibility and satisfaction survey with 40
socially isolated older adults (age $\geq$ 75) with MCI or normal
cognition to evaluate the chatbot and identify areas for future
improvement.} We will recruit participants to use our chatbot for
intervention and conduct a brief survey among them. Each participant
will receive 24 sessions in 6 weeks (4 sessions per week). Each session
will be an independent conversation with our chatbot for 15 minutes. A
staff member will do weekly 15-min check-in phone calls to assess the
emotional status and potential risks in using AI. We will collect both
objective data and subjective surveys to assess the feasibility and
satisfaction. (i) \textbf{Objective Data}. We will collect the adherence
rate, the portion of participants who completed over 80\% of sessions,
and the engagement degree, the ratio of words spoken by the user. (ii)
\textbf{Subjective data}. Each participant will do a survey after 6
weeks. The survey will focus on the participants' satisfaction with the
chatbot, including willingness to use the device in the future,
emotional influence, opinions on the quality of the chatbot, for
instance, if the chatbot presents empathy, and high fluency.

\textbf{Impact}. Our team is highly interdisciplinary, comprising a
statistician, a neurologist, and multiple computer scientists in trial
design/analytics, all with extensive experience in AD research. This R21
is especially timely, addressing the growing demand for complementary
alternatives to traditional human-led interventions. If feasible and
successful, it will greatly increase the accessibility of early dementia
intervention for socially isolated older adults.

\section{Significance}

\textbf{Alzheimer's Disease and related dementias (AD/RD) are difficult
to treat.} Although the FDA has approved several drugs to treat AD,
including drugs for mild AD, moderate to severe AD, and, more recently,
disease-modifying immunotherapy, they come with side effects and high
cost. Compounding these challenges, only a limited number of patients
are eligible for the new therapies. As millions of patients remain
without pharmaceutical treatment options, with estimates suggesting that
up to $\sim 40\%$ of AD risk may be attributable to modifiable
factors, early behavioral interventions represent a critical opportunity
to alter disease trajectories and reduce the growing burden of ADRD
\cite{livingston2020dementia}.

One well-documented risk factor of dementia is social isolation, marked
by limited social networks and support. Epidemiological research has
shown that social isolation raises the risk of cognitive decline and
dementia \cite{evans2019social,kallianpur2023weak,nasem2020social,penninkilampi2018association,poey2017social,shen2022associations}. The Lancet Commission on Dementia Prevention
reports that reducing social isolation could prevent 4\% of dementia
cases, surpassing the impacts of reducing physical inactivity and
diabetes, which are associated with 2\% and 1\% reductions, respectively
\cite{livingston2020dementia}. Thus, reducing social isolation could have a large impact on
reducing the prevalence of dementia. One recently completed intervention
study, which provided frequent social interactions through video chats,
recruiting socially isolated older subjects with mild cognitive
impairment (MCI, a precursor stage of dementia), showed a strong
efficacy (i.e., I-CONECT project (www.i-conect.org): NCT02871921). It
showed that frequent conversational interactions with human interviewers
could lead to enhanced cognitive functions \cite{dodge2024internet}.

\textbf{The high cost of human-led intervention limits its scalability.}
Though the finding could inspire future dementia prevention methods
based on conversational engagement, the high cost of hiring human
interviewers weekly could be cost-ineffective and non-sustainable for
many older adults. In 2022, 57.6 million Americans were age 65 and
older. Half of all older adults had less than \$29,740 (US dollars) in
yearly income from all sources \cite{pensionrights2023income}. It costs at least \$2,496 to
get a chat support service (according to the voice support salary in
2024 \cite{ziprecruiter2024chat}) for a twice-per-week service lasting one year. The high
cost could impede many older adults from accepting the method at the
risk of dementia.

\textbf{Intervention via Artificial Intelligence (AI) is scalable but
under development.} As the conversational AI (e.g., ChatGPT) emerges as
a powerful tool to respond to humans' prompts with in human-like natural
language, it is possible that the high cost of conversational engagement
could be lowered by replacing human interviewers with AI. According to
the latest price of state-of-the-art AI, OpenAI GPT-4o (in July 2024
\cite{openai2024pricing}), the cost for a twice-per-week service will be approximated as
\$7.68, only 0.3\% of the cost of the human service, suggesting a
cost-effective and affordable alternative for older adults.

Despite the low cost of conversational AI, its potential for
conversational engagement has not been proven yet. Once proven, an
AI-based chatbot may be used for dementia prevention in future clinical
trials. As a pilot study, this project will examine whether the
conversational AI (e.g., OpenAI GPT-4o) can simulate the interviewers'
language behaviors in the I-CONECT project (using the same picture
prompts and conversational styles from the efficacy-proven study) and
implement the same intervention strategies (stimulating cognitive
functions and sufficiently engaging interviewees) for future clinical
trials. Specifically, our chatbot will be designed to offer cognitively
demanding conversations, mimicking the I-CONECT trial's interactions
with human interviewers, with a user- and MCI-patient-friendly interface
that is friendly to older adults and ones with mild congitive
effectively engage them. The chatbot will ensure natural, effortless
conversations through an engaging vocal interface, accommodating older
adults with varying cognitive capacities. If successfully deployed, the
AD/RD-oriented chatbot can greatly improve the accessibility of early
dementia prevention for socially isolated older adults in terms of
accessibility and cost.

\section{INNOVATION}

\textbf{Therapy-aware AI chatbot for engaging conversational intervention.} We propose to develop the cost-effective AI-based
conversational intervention when a human-based one has been proven
helpful for dementia prevention \cite{dodge2024internet}, but could be too costly in
practice. The AI chatbot will be trained to implement the I-CONECT
protocol, delivering reminiscence therapy and vocabulary enhancement,
methods that benefit cognitive functions. In addition, the chatbot will
be optimized to align the I-CONECT interviewers' communication skills.

\textbf{Cost-effective development via data-driven simulation.} Iterating the design is costly due to getting feedback from human-based
evaluation. Our method simplifies the process by building virtual users
that simulate the language conversations in iterations with tested
chatbots. Using the virtual users, we establish a benchmark with novel
metrics to justify whether the chatbot can effectively engage older
adults. The benchmark can be used to guide further chatbot design for
older adults.

\textbf{User-friendly multi-modality interface minimizing the technology barrier for older adults.} Because older adults could be
unfamiliar with the emergent AI technology, they will be less willing to
engage with the AI interview, thereby reducing the effects of AI-based
intervention. Our interface is a software pre-installed on a mobile
tablet that is designed to be ready to use. The interface includes
multi-modality interactions: voice input and output (driven by language
modality) for conversation and generative facial expressions for
affective interactions. We will use a small AI model to synchronize the
voice, facial, and language interactions to provide human-like
experiences.

\section{Approach}

\textbf{Intervention Approach.} The recent research, an Internet-based conversational engagement
randomized controlled clinical trial (I-CONECT), evidenced that
stimulating frequent social interactions with trained interviewers via
the Internet/webcam could improve global cognitive functions and
episodic memory among socially isolated older subjects (aged 75 and
above) with mild cognitive impairment (MCI) \cite{dodge2024internet}. In this proposal
(AI-CONECT), we \textbf{hypothesize} that a chatbot can mimic a human
interviewer in these interactions and follow a similar intervention
strategy as the human interviewers did in the human-based I-CONECT
project. Based on I-CONECT, we propose principles of intervention and
verify the hypothesis through two-step evaluation: scalable virtual
testing and pilot human survey.

\textbf{Principles of Intervention Strategies}. The intervention
strategy addresses two main principles: (1)
\emph{\ul{Cognitively-Demanding Conversation Strategies}}: This includes
engaging users in novel chat experiences to stimulate their cognitive
functions, which may enhance brain connectivity and resist
neurodegeneration. Strategies of engaging includes providing visual
information, novel chat themes and cognitively-demanding topics to
stimulate users' cognitive activities. (2) \emph{\ul{User-Friendly
Device}}: Frequent engagement essentials the accessibility of the
service. Thus, we will design a user-friendly device that lowers the
barrier for technologically challenged older adults to use the service
at any time. Data privacy is also emphasized to build trust with users
and comply with regulations like HIPAA. Overall, these strategies aim to
provide cognitive benefits through interactive and engaging
conversations tailored to the needs and capabilities of socially
isolated older adults.

\textbf{Development Approach.} The key challenge is to implement the intervention strategies that have
been practiced in the I-CONECT trial. A system customizing AI and
validating compliance is essential before being used by humans. Thus,
our main technology includes three folds. All development or deployments
will be carried out in HIPAA-compliant servers to ensure data privacy.

(1) \textbf{Implementing Intervention Strategies via Customizing AI}: We
use the advanced Large Language Model (LLM) as the core of AI, which is
able to communicate with humans in natural language. We will customize
the state-of-the-art LLM (GPT-4o or its variants) to follow the
intervention strategies from the I-CONECT project. We will engineer
prompts that instruct LLMs to carry out predefined tasks, including
proposing chat themes and topics, and presenting topic-related images.
To enhance LLM's capability in engaging users in 30-min conversations,
we will improve the conversation engagement using heuristics and data
from the I-CONECT trials. On the one hand, we will consult interviewers
and related researchers (from Dr. Dodge's team) to provide heuristic
conversation strategies and encode the heuristics in the instructions
for LLMs. On the other hand, we will finetune the LLMs to learn the
spontaneous language patterns, for example, the active use of filler
words, from interviewers' speech in the I-CONECT conversation data. The
conversation data will be processed from audio recordings, and we will
develop speaker-aware Automatic Speech Recognition to extract
high-quality text data for LLMs finetuning. To avoid LLMs memorizing or
being biased toward users' private information, for example, personal
experiences that should not be shared with a third party, we will use
privacy-preserving machine learning in the process \cite{xiang2024guardagent}.

(2) \textbf{Conversation Simulation for Implementation Validation}: To
validate the implementation of the conversation strategies, we will
create virtual users that are LLMs finetuned on old adults' recorded
conversations and then simulate conversations with the socially-isolated
older adults to quantitatively evaluate the chatbot. The simulation will
learn the I-CONECT participants' language traits that may drive the
engagement designs. To provide a concrete simulation of older adults
with MCI, we will validate that the virtual users can generate similar
language symptoms that have been proven effective for MCI diagnosis
\cite{hoang2023subject}. For safety purposes, we will examine the biased content
generated by LLMs in the conversations. We will fix the problems by
creating a cleaned dataset and continually fine-tuning LLMs to forget
the contents \cite{liu2024rethinking}.

(3) \textbf{Device Development}: We will develop an old-adult-centered
interface device to facilitate the accessibility of our chatbot service.
We will minimize the operation complexities for older adults such that
older adults without computer-use experience can easily start a
conversation in natural language. The service will be available 24 hours
a day through voice, which is friendly for older adults with difficulty
typing. We will get feedback from survey participants on the
accessibility and improve the designs accordingly. Finally, we will
deliver an AI-empowered voice chatbot software that can be installed in
portable hardware, which implements I-CONECT conversation strategies and
attains high satisfaction from our survey participants.

\textbf{Feasibility Evaluation of the AI-based Implementation during
Development}. We will evaluate the chatbot in both virtual and human
evaluations to validate if the chatbot is able to implement the same
intervention strategies as the human interviewers. (1) \emph{\ul{Virtual
Testing}}. In the early stage of development, we adopt a cost-effective
method to evaluate the chatbot, where we simulate conversations between
the designed chatbot and a virtual user. The virtual user is constructed
as a digital simulation of a real old adult with or without mild
cognitive impairment. Extensive simulation enables scalable tests to
study how well the chatbot follows the intervention strategies. In the
simulation, we will use GPT-4o to automatically examine if the chatbot
will stick to the principles of intervention strategies to provide novel
chat experiences and visual stimulations (through external visual
tools).

\textbf{Approach for Feasibility Study.} We will test the chatbot on the stakeholders at the AITC and our
performance sites to get a feasibility survey that can guide the
improvement of the chatbot design. About 20 recruited older human
participants (age $\geq 75$) will be recruited to complete
feasibility/satisfaction surveys on their overall experience in terms of
adherence, happiness, emotional lifting, empathy, and trustworthiness
(e.g., reporting harmful or hateful biases). The survey will also
provide insights into how to design the chatbot to engage humans and
increase social interactions.

\textbf{Participant Recruitment}. We will recruit a total of \textbf{40
socially isolated older adults aged $\geq75$ years} from the MGH Memory
Division and the I-CONECT Foundation. The cohort will consist of two
balanced groups: \textbf{20 individuals with mild cognitive impairment
(MCI)} and \textbf{20 cognitively normal (NC)} participants.

Inclusion criteria:

\begin{enumerate}
\def\labelenumi{\arabic{enumi}.}
\item
  Aged 75+
\item
  Clinical diagnosis of MCI
\item
  Identified as socially isolated by one of:

  \begin{enumerate}
  \def\labelenumii{\alph{enumii}.}
  \item
    score $\leq 12$ on the 6-item Lubben Social Network Scale (LSNS-6)
    \cite{lubben2006performance}
  \item
    engages in conversations lasting 30 minutes or longer, no more than
    twice per week, per subject self-report
  \end{enumerate}
\item
  GDS-15 (15-item Geriatric Depression Scale) at/below 9 (not severely
  depressed) \cite{yesavage1982development}.
\item
  Answering "Often" to at least one question on the Hughes et al.
  Three-Item UCLA Loneliness Scale \cite{russell1978developing}.
\item
  Ability to understand the research consent form.
\item
  Cognitive status assessed using the Telephone Interview for Cognitive
  Status (TICS) \cite{fong2009telephone} should be MCI or NC, depending on the group.
\end{enumerate}

Exclusion criteria:

\begin{enumerate}
\def\labelenumi{\arabic{enumi}.}
\item
  Diagnosed with dementia such as AD, ischemic vascular dementia, normal
  pressure hydrocephalus, or Parkinson's disease.
\item
  Schizophrenia, or other major psychiatric disorder defined by DSM-IV
  criteria \cite{wilson1994special}.
\item
  Medications: Frequent use of high doses of analgesics; Use of sedative
  medications except for those used occasionally for sleep ($\leq2$ per
  week); Use of unstable dosing of Cholinesterase inhibitors (need to be
  stable dosing for 2 months).
\end{enumerate}

\section{DATA PROCESSING AND ANALYSES}

\textbf{Aim 1}: We will use I-CONECT recorded conversations to evaluate
and optimize the chatbot design for better implementation of the
I-CONECT protocol. For the device interface development, we will
organize discussions with clinicians and caregivers.

\textbf{Method: Processing I-CONECT Data and Constructing Digital
Twins.} We will use the existing I-CONECT audiovisual corpus of older
adult interviews (with and without mild cognitive impairment, MCI) to
design and optimize the AI-CONECT chatbot. All data will be processed on
HIPAA-compliant servers with identifiers removed.

\begin{enumerate}
\def\labelenumi{\arabic{enumi}.}
\item
  \textbf{Data preprocessing:} Audio will be transcribed using a
  speaker-aware ASR optimized for older adult speech, separating
  interviewer and participant utterances. Transcripts will be aligned
  with metadata, and quality checked manually and automatically.
\item
  \textbf{Annotation and feature extraction:} We will extract
  conversational and linguistic features (e.g., word counts, lexical
  diversity, reminiscence markers) from transcripts. A subset will be
  manually annotated for reminiscence and cognitively demanding prompts.
\item
  \textbf{Digital twin construction:} We will create LLM-based "digital
  twins" of older adult participants by fine-tuning models on aggregated
  participant data (not verbatim transcripts) to simulate their language
  patterns and engagement behaviors, reflecting differences between MCI
  and cognitively normal participants.
\item
  \textbf{Simulation-based evaluation:} The digital twins will simulate
  large-scale conversations with the AI-CONECT chatbot, enabling
  low-cost testing and controlled experimentation by varying chatbot
  strategies.
\item
  \textbf{Virtual benchmarks:} We will establish quantitative
  performance benchmarks against I-CONECT targets, including:

  \begin{itemize}
  \item
    \textbf{Engagement:} Participant word ratio (participant words/total
    words).
  \item
    \textbf{Reminiscence:} Reminiscence turn ratio (autobiographical
    turns/total participant turns).
  \item
    \textbf{Vocabulary enhancement:} Lexical richness indices (e.g.,
    type--token ratio).
  \end{itemize}
\end{enumerate}

\textbf{Iterative optimization:} Simulation results will guide
successive optimization cycles, revising chatbot parameters that fail to
meet engagement or reminiscence thresholds. Only validated chatbot
configurations will advance to human testing.

Collectively, Aim 1 establishes a rigorous, data-driven foundation for
translating the I-CONECT intervention protocol into a scalable, AI-based
conversational system while minimizing risk prior to human deployment.

\textbf{Aim 2}: We will conduct a survey to test whether AI is a
feasible solution as an alternative to a human-led one.

We will recruit a total of \textbf{40 socially isolated older adults
aged $\geq75$ years}, including \textbf{20 individuals with mild cognitive
impairment (MCI)} and \textbf{20 cognitively normal (NC)} participants.
This sample size is designed to support the \textbf{primary aims of
feasibility and user satisfaction}, rather than to test intervention
efficacy.

\textbf{Feasibility (Adherence).} The primary feasibility outcome is
adherence to the intervention, defined at the participant level as
completion of \textbf{$\geq80\%$ of scheduled AI-CONECT sessions}. A total
sample size of \textbf{N=40} allows estimation of the overall adherence
proportion with adequate precision for feasibility assessment. For
example, assuming an expected adherence rate of approximately
\textbf{80\%}, N=40 yields a 95\% confidence interval with a half-width
of approximately \textbf{$\pm 12\%$}, which is sufficient to determine
whether adherence meets the predefined feasibility threshold. This level
of precision is appropriate for an R21 feasibility study and supports
clear \textbf{go/no-go decisions} for future scale-up.\\
Including \textbf{20 MCI and 20 NC participants} ensures that
feasibility can be assessed across two cognitively distinct but relevant
subpopulations. With \textbf{N=20 per group}, adherence rates can be
estimated separately for MCI and NC participants with acceptable
precision for feasibility purposes, allowing identification of
group-specific implementation challenges (e.g., differential session
completion, technology burden, or need for support) without powering the
study for between-group efficacy comparisons.

\textbf{Satisfactory Outcome: Acceptability} of the AI-CONECT
intervention will be evaluated using the Client Satisfaction
Questionnaire--8 (CSQ-8), a validated measure of user satisfaction for
health and behavioral interventions, with total scores ranging from 8 to
32 and higher scores indicating greater satisfaction.
\textbf{Acceptability will be defined a priori using explicit,
quantitative criteria}, such that the intervention will be considered
well accepted if the \textbf{mean CSQ-8 score is $\geq24$}, \textbf{at least
70\% of participants achieve CSQ-8 scores $\geq24$}, and \textbf{no more than
15\% of participants score $\leq20$}, indicating the absence of systematic
dissatisfaction. A total sample size of \textbf{N=40 (20 MCI and 20
cognitively normal participants)} is sufficient for this purpose because
it provides adequate precision to evaluate these criteria without
hypothesis testing. Specifically, if approximately 70\% of participants
meet the satisfaction threshold (CSQ-8 $\geq24$), a sample of 40 yields a
95\% confidence interval with a half-width of approximately
\textbf{$\pm 14\%$}, allowing us to distinguish between unacceptably low
satisfaction (e.g., $<55$--$60\%$) and satisfaction levels
consistent with good acceptability. Similarly, with N=40, the mean CSQ-8
score can be estimated with sufficient precision to determine whether it
exceeds the prespecified benchmark of 24, while also allowing assessment
of score variability and detection of potential ceiling or floor
effects. The allocation of \textbf{20 participants per cognitive
subgroup} further allows acceptability to be summarized separately for
MCI and cognitively normal older adults, enabling identification of
subgroup-specific usability or burden concerns, while remaining
explicitly \textbf{not powered for formal between-group comparisons}.
This precision-based justification aligns with the feasibility and
acceptability aims of the study and supports clear go/no-go decisions
for refinement and future trials, consistent with the exploratory intent
of the R21 mechanism.

\textbf{Exploratory Outcomes.} Additional measures, including cognitive
assessments, will be collected for exploratory and descriptive purposes
only. Instead, these measures will be used to evaluate feasibility of
remote assessment, completion rates, and data quality, and to generate
preliminary signals to inform the design of future trials.

In summary, a total sample size of \textbf{20 MCI and 20 NC
participants} is well-suited to the goals of this R21 by providing
sufficient precision to evaluate adherence-based feasibility and user
satisfaction, while remaining consistent with the exploratory and
developmental intent of the mechanism and the ``clinical trial not
allowed'' designation.

\section{POTENTIAL RISKS AND ALTERNATIVE STRATEGIES}

The project
faces four main risks with corresponding alternative strategies:
\textbf{Risk 1} concerns the insufficient quality of I-CONECT audio data
for ASR and LLM fine-tuning, which will be mitigated by employing
advanced data augmentation and noise reduction, leveraging pre-trained
domain-specific models, and shifting emphasis to extensive prompt
engineering if necessary. \textbf{Risk 2} addresses the challenge of low
recruitment or poor adherence among older adults in the human survey
(Aim 2), which will be countered by utilizing existing collaborations
through the ADRC and AITC, simplifying the protocol (brief sessions,
provided tablet), and employing staged recruitment with early feedback.
\textbf{Risk 3} involves safety and trustworthiness concerns regarding
the LLM\textquotesingle s potential to generate harmful or non-factual
content, which will be managed through robust pre- and post-generation
content filtering, implementing limited, privacy-preserving memory
management, and integrating an in-app reporting mechanism for real-time
feedback. Finally, \textbf{Risk 4} is the difficulty in achieving
human-level conversational engagement, which will be addressed through
iterative fine-tuning guided by virtual testing metrics like "Word
Ratio" and digital twin satisfaction, post-survey algorithmic
adjustments based on human feedback on empathy and stimulation, and
potentially exploring a human-in-the-loop hybrid approach in future work
if AI-only engagement proves insufficient.

\section{IMPACT}

Our team is highly interdisciplinary, comprising a
statistician, a neurologist, and multiple computer scientists in trial
design/analytics, all with extensive experience in AD research. This R21
is especially timely, addressing the growing demand for complementary
alternatives to traditional human-led interventions. If feasible and
successful, it will greatly increase the accessibility of early dementia
intervention for socially isolated older adults.

\clearpage
\bibliographystyle{unsrtnat}
\bibliography{references}

\end{document}
