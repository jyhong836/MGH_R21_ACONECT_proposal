\textbf{Project Summary/Abstract}

Alzheimer's disease and related dementias (ADRD) are a growing public health burden, and limited social interaction is a modifiable risk factor for cognitive decline in older adults.
The Internet-based Conversational Engagement Clinical Trial (I-CONECT, NCT02871921) demonstrated that semi-structured, high-frequency conversational interactions with cognitive stimulation produced significant cognitive benefits. 
However, reliance on trained conversational staff limits scalability. This project will develop a Conversational AI with Multimodal Interaction (AIMI-CONECT), an artificial intelligence-driven conversational system designed to deliver the I-CONECT conversation protocol. 
The long-term objective is to enable scalable, affordable delivery of evidence-based conversational engagement for older adults with limited social interaction and to inform future, larger studies. 
The \textbf{specific aims} of the project are: (1) develop AIMI-CONECT to follow the efficacy-proven I-CONECT conversation protocol using multimodal interaction (including guided reminiscence with voice interaction, collaborative 3D memory scene recreation and visual affective responses), while incorporating methods to maintain protocol compliance and real-time safety monitoring; and (2) conduct a feasibility study with 40 older adults with limited social interaction age 75 years and older with mild cognitive impairment (MCI) or normal cognition, using four 15-minute sessions per week for six weeks and brief weekly staff check-ins, to assess emotional status and
potential risks or any technical challenges in using AI. The feasibility will be assessed through collecting recruitment ratio, adherence rate (e.g., the proportion completing $\ge80\%$ of sessions), participant's engagement level, and acceptability via client satisfactory surveys. Expected outcomes are a validated prototype, evidence supporting feasibility and user acceptance, and a clear path to refinement and large-scale studies. 
% This work is highly relevant to public health because 
The project addresses a key modifiable risk factor for dementia, a major challenge to public health, through a scalable approach targeting on older adults who lack access to frequent, human-delivered interventions.

\clearpage
\textbf{Project Narrative}

Dementia and cognitive decline impose a large public health burden, and limited social interaction increases this risk in older adults. Through system design, algorithm development and feasibility study, this project will develop AIMI-CONECT, an accessible conversational AI system with multimodal interaction that provides semi-structured conversational engagement at home for older adults with limited social interaction and mild cognitive impairment or normal cognition. 
The approach could expand access to evidence-based social engagement for broader populations at lower cost and with higher accessibility.
% The project could pave the way to expand access to evidence-based social engagement for communities that cannot sustain frequent human-delivered programs.

\clearpage
\textbf{Resource Sharing Plan(s)}

My institution Massachusetts General Hospital and I will adhere to the NIH Sharing Policies and Related Guidance on NIH-Funded Research Resources, including ``NIH Research Tools Policy'' and ``NIH Data Management and Sharing Policy.'' Following the characterization and peer-reviewed publications of the proposed project, related materials and data will be available to other investigators. The recipient investigators would provide written assurance that the data would only be used for research and not for commercial purposes. 

No proprietary materials are expected in this project. If any unique, non-proprietary resources are developed, they will be distributed to qualified investigators with standard data use and safety conditions in accordance with the NIH Principles and Guidelines document. ``Other Research Resources'', as available, will also be freely distributed upon request to qualified academic investigators for non-commercial research.


% \comm{Need to re-organize below.}
% \textbf{Model organisms:} Not applicable. This project does not develop model organisms.

% \textbf{Research tools and materials:} We will share the AIMI-CONECT software components, conversation prompts, survey instruments, codebooks, and analysis scripts. Software will be released in a public code repository under an open license and archived with a persistent identifier. Materials will be shared no later than the time of the first related publication.

% \textbf{Other resources:} No proprietary materials are expected. If any unique, non-proprietary resources are developed, they will be distributed to qualified investigators with standard data use and safety conditions.


\clearpage
\textbf{Data Management and Sharing Plan}

Scientific data from this project will be managed, preserved, and shared based on the below plan.

\textbf{Data type:} (1) \textit{Data that will be generated.} The project will collect conversation video/audio recording, conversation transcripts with names and direct identifiers removed, derived language and quantitative measures, survey responses, and system use logs from 20 MCI and 20 normal cognition participants. 
(2) \textit{Data that will be preserved and shared.}
Participant-level data described in 1 will be de-identified before submitting to repository and preserved through deposition of the data in a controlled access public repository.
The data collection system will be designed to minimize the collection of personally identifiable information (PII). Raw audio/video will not be shared. 
(3) \textit{Meta data and documents.} Submitted documentation will include study protocols, a table that explains each variable, and survey instruments. 

\textbf{Related tools, software, and code:} Data processing and analysis Python code will be shared in a public GitHub code repository, with clear instructions (including parameters) to reproduce reported results.
Tools used in the project will include open-source python software and libraries, and any custom tools will be shared with documentation in the code repository.

\textbf{Standards:} (1) Tabular data from demorgraphic and acceptability survey and other measurable outcomes (adherence rate, recruitment rate and engagement level) will be provided in non-proprietary formats such as comma-separated values (CSV). Variable names, definitions, and allowed values will be documented in a table that explains each variable. Where applicable, we will use common measures and reporting formats used in aging and cognitive health research.
(2) Audio data will be stored in WAV format, and video data in MP4 format. Transcripts will be stored in jsonl format.


\textbf{Data preservation, access, and timelines:} 
% (1) \textit{Repository to be used for data preservation and sharing}: De-identified datasets and documentation will be deposited in a public repository that provides long-term preservation, such as AD Data Initiative Repository.
% (2) \textit{How data will be findable and identifiable}: 
De-identified datasets and documentation will be deposited in a public repository that provides permanent identifiers and long-term preservation, such as AD Data Initiative Repository. 
The identifier will be included in the final publication. A data use agreement will be required to access the data. 
% (3) \textit{When and how long the data will be made available}: 
Data will be shared no later than the time of an associated publication, and will remain available for at least five years after release.
% \comm{check again.}

\textbf{Access, distribution, or reuse considerations:} We will maximize appropriate sharing while protecting participant privacy and confidentiality. 
(1) \textit{Factors affecting subsequent access, distribution, or reuse of scientific data}: The data will be collected with the informed consent: The dataset will be only used for research purposes and does not include the study of population origins or ancestry.
(2) \textit{Access Control}.
Data will be reviewed for the risk that someone could match the data back to a person. 
% If any dataset includes sensitive variables that could increase this risk, access will be controlled through a \textit{data use agreement} and review process. 
A data use agreement will be required to access the data with review process. 
(3) \textit{Privacy Protection}: Study data will be stored and processed within Massachusetts General Brigham (MGB) systems in accordance with the Health Insurance Portability and Accountability Act (HIPAA), encrypted in transit and at rest, and protected by secure authentication and periodic security review.

\textbf{Oversight of data management and sharing:} The Principal Investigator for the project, Dr. Junyuan Hong, the program manager, Ms. Cathrine Young, and a designated study coordinator, will oversee compliance with this plan, review data sharing readiness before release, and conduct compliance review annually.
The progress toward the plan's DMS activities will be included in reports submitted to the data repository officer.
At the project conclusion, the final progress report will summarize how the DMS objectives were fulfilled and provide links to the shared dataset(s).
