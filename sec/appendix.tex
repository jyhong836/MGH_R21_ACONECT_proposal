\section{Project Summary/Abstract}
Alzheimer's disease and related dementias (ADRD) are a growing public health burden, and social isolation is a modifiable risk factor for cognitive decline in older adults. Evidence from the Internet-based Conversational Engagement Clinical Trial (I-CONECT) shows that structured, frequent conversation can help, but delivery by trained staff is difficult to scale. This project will develop a Conversational AI with Multimodal Interaction (AIMI-CONECT), an artificial intelligence (AI) conversational system designed to deliver the I-CONECT conversation guide with safe, consistent, and accessible sessions. The long-term objective is to enable scalable, affordable delivery of evidence-based conversational engagement for socially isolated older adults and to inform future, larger studies.
The specific aims are: (1) develop AIMI-CONECT to follow the I-CONECT conversation flow using a voice-forward interface and multimodal interaction (including guided reminiscence with optional collaborative 3D memory scene recreation and visual affective responses), while incorporating methods to maintain protocol compliance and real-time safety monitoring for distress, misinformation, and financial risk; and (2) conduct a feasibility study with 40 socially isolated older adults age 75 years and older with mild cognitive impairment (MCI) or normal cognition, using four 15-minute sessions per week for six weeks and brief weekly staff check-ins, to assess feasibility, acceptability, and safety. We will assess adherence (e.g., the proportion completing $\geq$80\% of sessions), engagement, usability, and acceptability using use logs and surveys. Expected outcomes are a validated prototype, evidence supporting feasibility and user acceptance, and a clear path to refinement and larger studies. This work is relevant to public health because it addresses a major modifiable risk factor for dementia with an approach that can reach older adults who lack access to frequent human-delivered programs.

\section{Project Narrative}
Dementia and cognitive decline impose a large public health burden, and social isolation increases risk in older adults. This project will create AIMI-CONECT, a safe and accessible artificial intelligence (AI) conversational system that provides structured conversational engagement at home for socially isolated older adults using a voice-forward, multimodal interface. The approach could expand access to evidence-based social engagement for communities that cannot sustain frequent human-delivered programs, while prioritizing protocol compliance, safety monitoring, and privacy.

\section{Resource Sharing Plan(s)}
\textbf{Model organisms:} Not applicable. This project does not develop model organisms.

\textbf{Research tools and materials:} We will share the AIMI-CONECT software components, conversation prompts, survey instruments, codebooks, and analysis scripts. Software will be released in a public code repository under an open license and archived with a persistent identifier. Materials will be shared no later than the time of the first related publication or the end of the project period, whichever comes first.

\textbf{Other resources:} No proprietary materials are expected. If any unique, non-proprietary resources are developed, they will be distributed to qualified investigators with standard data use and safety conditions.

\section{Data Sharing Plan}
This Data Management and Sharing (DMS) Plan describes how scientific data from this project will be managed, preserved, and shared.

\textbf{Data type:} We will generate conversation transcripts with names and direct identifiers removed, derived language and engagement measures, survey responses, and system use logs. The system will be designed to minimize the collection of personally identifiable information (PII). Documentation will include study protocols, a table that explains each variable, and survey instruments. Raw audio will not be shared unless consent and privacy protections allow it.

\textbf{Related tools, software, and code:} Data processing and analysis scripts will be shared in the same public code repository as the project software, with clear instructions to reproduce reported results.

\textbf{Standards:} Tabular data will be provided in non-proprietary formats such as comma-separated values (CSV). Variable names, definitions, and allowed values will be documented in a table that explains each variable. Where applicable, we will use common measures and reporting formats used in aging and cognitive health research.

\textbf{Data preservation, access, and timelines:} Datasets and documentation will be deposited in a public repository that provides permanent identifiers and long-term preservation, such as an institutional repository or the Open Science Framework. Data will be shared no later than the time of an associated publication or the end of the project period, whichever comes first, and will remain available for at least five years after release.

\textbf{Access, distribution, or reuse considerations:} We will maximize appropriate sharing while protecting participant privacy and confidentiality. Data will be reviewed for the risk that someone could match the data back to a person. If any dataset includes sensitive variables that could increase this risk, access will be controlled through a data use agreement and review process. Use will be limited to research purposes consistent with consent and applicable regulations. Study data will be stored and processed within Massachusetts General Hospital (MGH) systems in accordance with the Health Insurance Portability and Accountability Act (HIPAA), encrypted in transit and at rest, and protected by secure authentication and periodic security review.

\textbf{Oversight of data management and sharing:} The Principal Investigator and a designated data manager will oversee compliance with this plan, review data sharing readiness before release, and conduct compliance checks at least every six months.
