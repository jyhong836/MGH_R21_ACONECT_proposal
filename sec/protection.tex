\clearpage
\section{Protection And Monitoring Plans}

\clearpage
\textbf{Protection of Human Subjects}

% TODO - Directly modify the content based on examples/protection_of_human_subject.docx
% Checklist:
% [x] Risks to human subjects
% [x] Adequacy of protections (including consent and privacy)
% [x] Potential benefits
% [x] Importance of knowledge to be gained

\textbf{1. Risks to Human Subjects}

\textit{a. Human Subjects involvement, characteristics, and design.}
This project includes voluntary participation of older adults with limited social interaction, age 75 years or older, with mild cognitive impairment (MCI) or normal cognition (NC). Participants will be recruited from the Memory Division at Massachusetts General Brigham (MGB) and the Massachusetts Alzheimer's Disease Research Center (MADRC). The study is a feasibility study of Conversational Artificial Intelligence with Multimodal Interaction (AIMI-CONECT), with four 15-minute sessions per week for six weeks and weekly staff check-in calls. The intervention is behavioral and conversational, with no drug administration and no invasive procedures.

\textit{b. Study procedures, materials, and potential risks.}
Study procedures include conversational sessions with AIMI-CONECT, brief weekly check-ins, and collection of study data including conversation audio and video records, de-identified transcripts, derived language features, survey responses, and system logs. The main risks are expected to be low and include emotional discomfort during conversations, fatigue or frustration from repeated sessions, exposure to unsafe or inaccurate model-generated content, technology burden, and loss of confidentiality if identifiable data are improperly disclosed.

\textit{c. Protections against risks.}
The protocol includes real-time safety guardrails that monitor for critical situations such as negative affect, suicidal language, misinformation, and financial risk content, with immediate alerts to study staff for follow-up. Staff will conduct weekly check-ins to assess emotional status, potential risks, and technical barriers, and participants may pause or stop sessions at any time. Data protection includes informed consent, minimization of personally identifiable information, de-identification for analysis, encrypted storage and transfer, and access control within Health Insurance Portability and Accountability Act (HIPAA)-compliant MGB systems. All study procedures will start only after Institutional Review Board (IRB) approval, and all key personnel will complete required human-subject and privacy training.

\textbf{2. Adequacy of Protection Against Risks}

\textit{a. Recruitment and informed consent.}
Eligible participants will be identified using the study inclusion and exclusion criteria defined in the research strategy. The study team will review the consent form in plain language, explain study purpose, procedures, risks, data use, and alternatives, and allow adequate time for questions before any research procedures begin. Written informed consent will be obtained before participation. Only participants with the ability to provide informed consent will be enrolled.

\textit{b. Risk management and confidentiality safeguards.}
Safety monitoring combines automated in-conversation monitoring with staff review and follow-up. Any event that suggests participant distress, safety concern, or technology-related burden will be documented, assessed, and managed by the study team under the approved protocol. Confidentiality will be protected through role-based data access, secure storage, and separation of direct identifiers from analytic datasets. Results will be reported in aggregate form, and no participant-identifying information will be disclosed in publications.

\textbf{3. Potential Benefits of the Proposed Research to Human Subjects and Others}

Participants may benefit from structured conversational engagement, increased opportunities for social interaction, and a low-burden interface designed for older adults. The study may also provide practical insight into how to design safer and more acceptable artificial intelligence tools for older adults with limited social interaction.

\textbf{4. Importance of Knowledge to be Gained}

The proposed research will generate evidence on the feasibility, acceptability, and implementation safety of an artificial intelligence-delivered conversational intervention adapted from the Internet-based Conversational Engagement Clinical Trial (I-CONECT). This knowledge is important for developing scalable, accessible approaches to support older adults who have limited access to frequent human-delivered conversational engagement.


\clearpage
\textbf{Data and Safety Monitoring Plan}


% TODO - Check below requirement and write the section.
% For any proposed clinical trial, NIH requires a data and safety monitoring plan (DSMP) that is
% commensurate with the risks of the trial, its size, and its complexity. Provide a description of the
% DSMP, including:
% l Indicate how many people and what type of entity will provide the monitoring. Include
% such details as whether a single person, multiple people, or a data safety monitoring
% board will provide monitoring. Also indicate what type of entity will provide the
% monitoring (e.g., PI, Independent Safety Monitor/Designated Medical Monitor,
% Independent Monitoring Committee, Safety Monitoring Committee, Data and Safety
% Monitoring Board, etc.).
% l The overall framework for safety monitoring and what information will be monitored.
% l The frequency of monitoring, including any plans for interim analysis and stopping rules
% (if applicable).
% l The process by which Adverse Events (AEs), including Serious Adverse Events (SAEs) such
% as deaths, hospitalizations, and life threatening events and Unanticipated Problems (UPs),
% will be managed and reported, as required, to the IRB, the person or group responsible for
% monitoring, the awarding IC and the Food and Drug Administration.
% l The individual(s) or group that will be responsible for trial monitoring and advising the
% appointing entity. Because the DSMP will depend on potential risks, complexity, and the
% nature of the trial, a number of options for monitoring are possible. These include, but are
% not limited to, monitoring by a:
% o PI: While the PI must ensure that the trial is conducted according to the
% approved protocol, in some cases (e.g., low risk trials, not blinded), it may be
% acceptable for the PI to also be responsible for carrying out the DSMP.
% o Independent safety monitor/designated medical monitor: a physician or other expert
% who is independent of the study.
% o Independent Monitoring Committee or Safety Monitoring Committee: a small group
% of independent experts.
% o Data and Safety Monitoring Board (DSMB): a formal independent board of experts
% including investigators and biostatisticians. NIH requires the establishment of DSMBs
% for multi-site clinical trials involving interventions that entail potential risk to the
% participants, and generally, for all Phase III clinical trials, although Phase I and Phase II
% clinical trials may also need DSMBs. If a DSMB is used, please describe the general
% composition of the Board without naming specific individuals.
% Checklist:
% [x] Monitoring entity and responsible individuals
% [x] Safety monitoring framework and monitored information
% [x] Monitoring frequency and stopping approach
% [x] AE/SAE/UP management and reporting pathway

This Data and Safety Monitoring Plan (DSMP) is designed for a low-risk, single-site feasibility study with behavioral conversational sessions. A Data and Safety Monitoring Board (DSMB) is not planned for this study design. Monitoring will be conducted by a multi-person team led by the Program Director/Principal Investigator (PI), with operational support from the program manager and designated study coordinator. The PI is responsible for overall study monitoring decisions and for advising the appointing institution on study continuation, protocol modification, or temporary pause when safety concerns arise.

The monitoring framework integrates automated and human review. Real-time guardrails in the conversational system monitor protocol drift and safety signals, including negative affect, suicidal language, misinformation, and financial risk content. Study staff review safety alerts, weekly check-in findings, session adherence patterns, participant complaints, and technology incidents. The team also monitors confidentiality events, data security issues, and any event that may increase risk to participants.

Monitoring occurs continuously at the session level through real-time guardrails and through regular staff review of accumulated safety and protocol data. Weekly participant check-ins provide an additional safety layer outside session time. Because this project is a feasibility study and does not test efficacy outcomes, no formal interim efficacy analysis is planned. Stopping guidance is risk-based: if new or worsening safety concerns are identified, the PI may pause affected sessions or broader study activities pending review and corrective action.

Adverse Events (AEs), Serious Adverse Events (SAEs), and Unanticipated Problems (UPs) will be documented in the study records, reviewed by the PI, and managed under the approved protocol and institutional policy. Events that require expedited reporting will be reported to the IRB, the National Institutes of Health (NIH) awarding Institute or Center (IC), and the monitoring entity within required timelines. Food and Drug Administration (FDA) reporting is not expected for this behavioral study; if reporting requirements apply, reporting will follow all applicable federal and institutional requirements.


\clearpage
\textbf{Overall Structure of the Study Team}

% TODO - Check below requirement and write the section. You can modify from examples/Overall structure of the study team.docx
% Provide a brief overview of the organizational/administrative structure and function of the study
% team, particularly the administrative sites, data coordinating sites, enrollment/participating sites,
% and any separate laboratory or testing centers. The attachment may include information on study
% team composition and key roles (e.g., medical monitor, data coordinating center), the governance
% of the study, and a description of how study decisions and progress are communicated and
% reported.
% Note: Do not include study team members’ individual professional experiences (i.e., Biographical
% Sketch Common Form and NIH Biographical Sketch Supplement information).
% Checklist:
% [x] Administrative, enrollment, and data-coordination structure
% [x] Key team roles and governance model
% [x] Communication and reporting workflow

This application leverages the expertise, infrastructure, and resources of a multidisciplinary team at Massachusetts General Hospital (MGH), within Massachusetts General Brigham (MGB). The team includes the Principal Investigator (PI), Co-Investigators (Co-Is), data and operations staff, and consultants. Administrative oversight and data coordination are managed centrally at MGH. Enrollment and participant-facing activities are conducted through the Memory Division at MGB and the Massachusetts Alzheimer's Disease Research Center (MADRC), with weekly check-in calls coordinated with Internet-based Conversational Engagement Clinical Trial (I-CONECT) Foundation staff. No separate laboratory or testing center is planned for this feasibility study.

\textbf{\underline{Principal Investigators and Co-Investigators}}

\textbf{Dr. Junyuan Hong (Principal Investigator)} will lead all aspects of project implementation, including regulatory coordination, data management oversight, secure system operations, analysis planning, and dissemination activities. He will oversee Institutional Review Board (IRB) preparation, ensure secure and reliable conversational system operations, and coordinate integration across technical, statistical, and operational workstreams.

\textbf{Dr. Hiroko H. Dodge (Co-Investigator)} will guide adaptation of the I-CONECT conversational protocol to ensure protocol fidelity and alignment with the target older-adult population. She will support study design, protocol implementation oversight, interpretation of feasibility findings, and manuscript development.


\textbf{\underline{Data Analysis Team}}

\textbf{Dr. Chao-Yi Wu (Co-Investigator)} will serve as the blinded lead statistician and will support analysis design, feasibility analyses, and reporting.  She will also work on preparing manuscripts and disseminating the study results. 

\textbf{Dr. Liu Chen (Data Manager)} will support the data analysis pipeline, including automated speech transcription for study conversations, language-feature extraction, and integration of remote assessment and satisfaction-survey workflows. He will also support the Research Electronic Data Capture (REDCap) data capture workflow and data preparation for analysis and dissemination.
Liu will manage the hardware and software systems utilized for the project (i.e. creating a Redcap database platform, videophones, statistical software, data processing, and workflow processes), monitoring recruitment and quality-control data entry for all patient visits and generating a weekly report for recruitment, randomization and adverse events. 
% The data manager will also serve as an unblinded statistician, generating descriptive statistics/tables and recruitment flows for the DSBM meeting. 
He or she will manage the initial deployment, storage and maintenance, user access, training and technical troubleshooting. He or she will ensure data integrity, and appropriate and secure data storage. 
She will also work on preparing manuscripts and disseminating the study results. 

\textbf{\underline{Study Staff}}

\textbf{Ms. Cathrine Young (Program Manager)} will oversee day-to-day operations, regulatory documentation, recruitment monitoring, and coordination across study personnel. She will supervise data operations and coordinate participant-support workflows with I-CONECT Foundation staff. 

\textbf{Study Coordinator (to be appointed)} will manage recruitment, screening, scheduling, assessment calls, participant reimbursement workflows, data entry, and coordination of weekly participant check-ins.

\textbf{\underline{Significant Others}}

\textbf{Dr. Zhangyang ``Atlas'' Wang} will advise on generative model design, technical feasibility, and system optimization for robustness and efficiency.

\textbf{Dr. Tianlong Chen} will advise on social-robot integration, hardware optimization, and hardware-aware artificial intelligence deployment. 

\textbf{I-CONECT Foundation Staff} will support weekly check-in calls and communication pathways for participant support and escalation when concerns arise.

The PI, Co-Is, and study staff will meet regularly to review recruitment, adherence, safety alerts, protocol compliance, and data quality. Operational decisions and action items will be documented by the central administrative team, and required reports will be submitted through institutional and sponsor reporting pathways.
