\clearpage
\section{Study Population Characteristics}

\clearpage
\textbf{Inclusion of Individuals Across the Lifespan}

% TODO - Check below requirement and write the section.
%  Content:
%     Discuss each of the points listed below. Also include any additional information requested in the
%     NOFO.
%     You will also have to complete an Inclusion Enrollment Report (IER). Note that you may need to
%     include multiple IERs for each study. Refer to the instructions for the IER below for more
%     information.
% Inclusion of Individuals Across the Lifespan:
%     For the purposes of the Inclusion of Individuals Across the Lifespan, exclusion of any specific age
%     or age range group (e.g., children or older adults) should be justified in this section. In addition,
%     address the following points:
%      * Individuals of all ages are expected to be included in all NIH-defined clinical research
%     unless there are scientific or ethical reasons not to include them. Discuss whether
%     individuals will be excluded based on age and provide a rationale for the minimum and
%     maximum age of study participants, if applicable. Additionally, if individuals will be
%     excluded based on age, provide a scientific or ethical rationale for their exclusion. See the
%     NIH Policy and Guidelines on the Inclusion of Individuals Across the Lifespan as
%     Participants in Research Involving Human Subjects for additional information about
%     circumstances that may justify the exclusion of individuals based on age.
%      * Include a description of the expertise of the investigative team for working with
%     individuals of the ages included, the appropriateness of the available facilities to
%     accommodate individuals in the included age range, and how the age distribution of
%     participants will contribute to a meaningful analysis relative to the purpose of the study.
%     When children are involved in research, the policies under HHS’ 45 CFR 46, Subpart D - Additional
%     Protections for Children Involved as Subjects in Research apply and must be addressed in the
%     Protection of Human Subjects attachment.
% Existing Datasets or Resources. If you will use an existing dataset, resource, or samples that may
%     have been collected as part of a different study, you must address   inclusion, following the
%     instructions above. Generally, you must provide details about the sex, race, and ethnicity of the
%     existing dataset/resource and justify the details as appropriate to the scientific goals of the
%     proposed study.
% Checklist:
% [x] State age inclusion/exclusion rationale
% [x] Describe team and facility readiness for the included age range
% [x] Explain contribution of age distribution to meaningful analysis
% [x] Address existing dataset/resource inclusion details

This feasibility study will enroll older adults aged 75 years and older. The lower age boundary is intentional and is consistent with the established I-CONECT operational criteria used in prior work with socially isolated older adults at elevated risk for cognitive decline. We will not exclude participants on the basis of advanced age alone; instead, eligibility will be based on cognitive status, social isolation criteria, safety screening, and ability to provide informed consent. Because the scientific objective is to evaluate feasibility of an artificial intelligence-delivered conversational protocol in this high-risk older population, excluding younger age groups is a scientific design choice rather than a convenience-based restriction.

The investigative team includes experts in neurology, gerontology, biostatistics, and computer science with prior experience conducting conversational intervention studies in older adults. Recruitment and study procedures will be supported through the Memory Division at Mass General Brigham (MGB) and the Massachusetts Alzheimer's Disease Research Center (MADRC), with staff oversight for weekly emotional status and safety check-ins during participation. The available facilities and staffing structure are appropriate for older adults with mild cognitive impairment (MCI) and normal cognition (NC), including those with limited prior technology experience.

The planned age distribution supports meaningful analysis because the target population for this project is older adults with limited social interaction, the group for whom scalable home-based delivery is most needed. The balanced enrollment target across cognitive strata (MCI and NC) will support feasibility interpretation across two clinically relevant subpopulations while remaining within the scope of this feasibility study.

For existing resources, we will use prior I-CONECT conversation data for model adaptation and protocol translation. In Table 1 of the published I-CONECT topline report, the randomized sample size was 186, including 130 women (69.9\%) and 56 men (30.1\%); race was reported as 149 White participants (80.1\%) and 37 participants who were not White (19.9\%). Ethnicity categories were not reported separately in that table. We will account for these characteristics when interpreting generalizability and will report prospective enrollment through the Inclusion Enrollment Report (IER), with separate report entries prepared as needed for distinct study components.


\clearpage
\textbf{Inclusion of Women and Members of Racial and / or Ethnic Minority Groups}
% TODO - Check below requirement and write the section.
% Discuss each of the points listed below and include any additional information requested in the
% NOFO.
% You will also have to complete an Inclusion Enrollment Report (IER). Note that you may need to
% include multiple IERs for each study. Refer to the instructions for the IER below for more
% information.
% Inclusion of Women and Members of Racial and / or Ethnic Minority Groups
% Address the following points:
%  * Describe the planned distribution of subjects by sex, race, and ethnicity.
%  * Describe the rationale for selection of sex, racial, and ethnic group members in terms of
% the scientific objectives and proposed study design. The description may include, but is
% not limited to, information on the population characteristics of the disease or condition
% under study.
%  * Describe proposed outreach programs for recruiting sex, racial, and ethnic group
% members.
%  * Inclusion and Excluded Groups: Provide a reason for limiting inclusion of any group by
% sex, race, and/or ethnicity. In general, the cost of recruiting certain groups and/or
% geographic location alone are not acceptable reasons for exclusion of particular groups.
% See the Inclusion of Women and Members of Racial and/or Ethnic Minority Groups in
% Clinical Research for more information.
% Existing Datasets or Resources. If you will use an existing dataset, resource, or samples that may
% have been collected as part of a different study, you must address inclusion, following the
% instructions above. Generally, you must provide details about the sex, race, and ethnicity of the
% existing dataset/resource and justify the details as appropriate to the scientific goals of the
% proposed study
% Checklist:
% [x] Planned distribution by sex, race, and ethnicity
% [x] Rationale for distribution aligned with study objectives
% [x] Outreach/recruitment approach for inclusive enrollment
% [x] Explicit statement on inclusion and non-exclusion by sex/race/ethnicity
% [x] Existing dataset/resource demographics and justification

The planned feasibility cohort includes 40 participants, with a 1:1 split between mild cognitive impairment (MCI) and normal cognition (NC). For sex, we will follow the ratio shown in the provided Inclusion Enrollment Report example (49:41 female-to-male), corresponding to approximately 22 women and 18 men in a cohort of 40. For race, we will follow the confirmed example ratio of 80\% White, 10\% Black or African American, and 10\% Asian, corresponding to 32, 4, and 4 participants, respectively. For ethnicity, we will plan for 10\% Hispanic or Latino (4 participants) and 90\% not Hispanic or Latino (36 participants).

This planned distribution is designed to support feasibility estimation in a cohort that is consistent with prior recruitment patterns while still preserving inclusive eligibility. The scientific objective of this proposal is to assess feasibility, acceptability, and implementation risks of the intervention platform, not to estimate differential effectiveness by demographic subgroup. Even so, sex, race, and ethnicity distributions will be tracked prospectively and reported through the Inclusion Enrollment Report.

Recruitment will occur through referral pathways and outreach associated with the MGB Memory Division and MADRC. Outreach materials and consent communication will use clear language, and staff will monitor enrollment continuously so recruitment efforts can be adjusted if underrepresentation emerges in any sex, racial, or ethnic group. There is no protocol-based exclusion by sex, race, or ethnicity.

For existing resources, we will use prior I-CONECT data to inform model and protocol development. The published Table 1 for that dataset reports sex and race composition (130 women, 56 men; 149 White participants and 37 participants who were not White, total N=186), while ethnicity categories are not separately reported in that table. These limits are acknowledged, and the prospective feasibility cohort will include complete sex, race, and ethnicity reporting in its Inclusion Enrollment Report.

\clearpage
\textbf{Recruitment and Retention Plan}

% TODO - Check below requirement and write the section.
% Describe how you will recruit and retain participants in your study. You should address both
% planned recruitment activities as well as proposed engagement strategies for retention.
% Checklist:
% [x] Planned recruitment activities
% [x] Retention and engagement strategies
% [x] Consistency with safety monitoring and feasibility outcomes

Participants will be recruited from the MGB Memory Division and MADRC using the inclusion and exclusion framework described in the Research Strategy. Recruitment will focus on older adults aged 75 years and older with limited social interaction and either mild cognitive impairment or normal cognition, with a planned total of 40 participants and balanced cognitive subgroup enrollment. Screening and consent procedures will emphasize clear communication, privacy protections, and participant understanding of study expectations.

Retention will be supported through a low-burden delivery model that uses a ready-to-use tablet interface and voice-forward interaction so participants can engage without extensive technical training. Each participant will complete four 15-minute sessions per week for six weeks (24 total sessions), and staff will provide weekly 10-minute check-in phone calls to monitor emotional status, potential safety risks, and technical challenges. If barriers arise, staff will provide prompt troubleshooting and practical support to maintain participation.

Engagement strategies are designed to sustain participation and conversational quality, including structured conversation themes, image-supported prompts, and adaptive interaction flow tailored to older adults. Retention and feasibility will be monitored through recruitment ratio, adherence rate (including completion of at least 80\% of scheduled sessions), user engagement metrics, and participant satisfaction reporting.

\clearpage
\textbf{Study Timeline}

% TODO - Check below requirement and write the section. Note our study is 2-year.
% Provide a description or diagram describing the study timeline. The timeline should be general
% (e.g., "one year after notice of award"), and should not include specific dates.
% Note: Additional milestones or timelines may be requested as just-in-time information or post-award. 
% Checklist:
% [x] General 2-year timeline without specific dates
% [x] Narrative plus timeline table
% [x] Alignment with feasibility and safety workflow

This project will be completed over two years with a month-wise schedule that starts recruitment-related work earlier. Development of Conversational Artificial Intelligence with Multimodal Interaction for I-CONECT (AIMI-CONECT) will be completed within the first 6 months. Preparation for the single-arm human feasibility study will begin in Month 3 and run in parallel with final development activities so recruitment can begin as soon as development is ready.

\begin{table}[h]
\centering
\begin{tabular}{p{0.24\linewidth} p{0.70\linewidth}}
    \toprule
\textbf{Period} & \textbf{Major activities} \\
\midrule
Month 1--6 & AI protocol translation, multimodal system integration, safety/compliance workflow buildout, internal testing, and readiness verification. \\
Month 3--6 & Human-study preparation in parallel with development, including study operations setup, staff workflow preparation, recruitment material finalization, and coordination with MGB and MADRC. \\
Month 6 onward & Start recruitment, screening, and rolling enrollment for the single-arm feasibility cohort. \\
Month 6 onward through Year 2 & Intervention delivery (four 15-minute sessions per week for six weeks per participant with weekly staff check-ins), ongoing safety and adherence monitoring, feasibility and acceptability assessment, and iterative refinement planning. \\
\bottomrule
\end{tabular}
\end{table}

Additional milestones and operational detail will be provided as just-in-time or post-award information if requested by the sponsor.
