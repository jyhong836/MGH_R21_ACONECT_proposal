\textbf{Specific Aims}

\textbf{Significance}. Alzheimer's disease and related dementias (ADRD)
remain a major and growing public health burden, and the disease's
multifactorial biology continues to outpace the impact of available
therapies\cite{zhang2024recent}. Current U.S. Food and Drug Administration-approved
symptomatic treatments offer modest benefits, while newer disease-modifying anti-amyloid immunotherapies are
limited to biomarker-confirmed early Alzheimer's disease and are
accompanied by restricted patient eligibility and significant treatment
burden with cost-effectiveness concerns\cite{mangalagiu2025pharmacological}. 
Thus, it is important to develop accessible and affordable prevention strategies early in the disease course.

\textbf{Critical Barriers.} Though social isolation is a recognized, modifiable
risk factor for cognitive decline and dementia in older adults
\cite{evans2019social,kallianpur2023weak,nasem2020social,penninkilampi2018association,poey2017social,shen2022associations}, effective interventions that can deliver sustained, high-fidelity social engagement at the population scale remain limited.
The I-CONECT clinical trial (NCT02871921) provides evidence that
semi-structured conversational engagement with cognitive stimulation can mitigate social isolation and cognitive decline among socially isolated older adults with Mild Cognitive Impairment (MCI) \cite{dodge2024internet}. Yet, human-delivered programs are constrained by workforce availability and high cost, rendering a persistent barrier to broad dissemination.

\textbf{Solutions.} To enhance the accessibility and affordability of
the I-CONECT intervention, we propose to develop an AI-led
conversational intervention (AI-CONECT) by leveraging advanced
Generative AI to simulate the human-delivered I-CONECT conversational
intervention. Following the I-CONECT intervention, the AI chatbot will
provide similar conversation strategies to engage participants and stimulate their cognitive functions. 
% The chatbot will be installed with an multi-modality interface that minimizes the gap to human interactions by using real-time voice input/output and visual emotion responses.
Our preliminary result\cite{hong2024aconect} has shown that Large Language Models (LLMs) can be customized to increase participants' (with MCI) engagement in
simulated conversations. However, the system was under-optimized for engaging older adults: limited number of participants, limited modality or device adaptation for a technique-unfamiliar population. Moreover, the
feasibility of the AI-led intervention has not been evaluated among the
targeted population yet. Thus, our specific aims in this proposal are:

\textbf{Aim 1: Develop a user-friendly multi-modality social chatbot
system (AI-CONECT) for executing the I-CONECT intervention.} We will
develop the chatbot for two design goals. (i) \textbf{Protocol
compliance}: We will use an agentic design to enable the AI chatbot to
follow the I-CONECT protocol: execute semi-structured conversations, engage participants in reminiscence and vocabulary enhancement. (ii) \textbf{User friendliness}: To reduce the use
barrier for older adults, we will integrate real-time voice input/output with visual
emotion-responsive interaction\cite{zhao2025transferring}. The whole system will be deployed on an older-adult-friendly tablet paired with an emotion-aware social robot
that provides real-time affective feedback. 
(iii) \textbf{Safety}: Limited awareness of AI risks may place older adults at higher risk when using AI systems. We will adopt real-time guardrails on the conversation to
monitor critical situations like low emotion, suicide, misinformation,
or financial risks\cite{xiang2024guardagent}.

\textbf{Aim 2: Conduct a feasibility and satisfaction survey with 40
socially isolated older adults (age $\geq$ 75) with MCI or normal
cognition to evaluate the chatbot and identify areas for future
improvement.} We will recruit participants to use our chatbot for
intervention and conduct a brief survey among them. Each participant
will receive 24 sessions in 6 weeks (4 sessions per week). Each session
will be an independent conversation with our chatbot for 15 minutes. A
staff member will do weekly 15-min check-in phone calls to assess the
emotional status and potential risks in using AI. We will collect both
objective data and subjective surveys to assess the feasibility and
satisfaction. (i) \textbf{Adherence}. We will collect the adherence
rate, the portion of participants who completed over 80\% of sessions,
and the engagement degree, the ratio of words spoken by the user. (ii)
\textbf{Satisfaction}. Each participant will do a survey after 6
weeks. The survey will focus on the participants' satisfaction with the
chatbot, including willingness to use the device in the future,
emotional influence, opinions on the quality of the chatbot, for
instance, if the chatbot presents empathy, and high fluency.

\textbf{Impact}. Our team is highly interdisciplinary, comprising a
statistician, a neurologist, and computer scientists in trial
design/analytics, all with extensive experience in AD research. This R21
is especially timely, addressing the growing demand for complementary
alternatives to traditional human-led interventions. If feasible and
successful, it will greatly increase the scalability of early dementia
intervention for socially isolated older adults.


\cleardoublepage
\section{Significance}

% \textbf{ADRD is prevalent but remains difficult to treat.}
ADRD is difficult to treat, and available pharmacologic options are limited by side effects, high cost, and restricted eligibility for newer disease-modifying
therapies\cite{zhang2024recent}. As millions of patients remain without accessible treatment
options, and with estimates suggesting that up to $\sim 40\%$ of AD risk
may be attributable to modifiable factors, scalable behavioral
interventions represent a critical opportunity to alter trajectories and
reduce the growing burden of ADRD \cite{livingston2020dementia}. 
% Social isolation is a well-documented, potentially modifiable risk factor:
Fir example, epidemiological research has repeatedly associated social isolation with
greater risk of cognitive decline and dementia
\cite{evans2019social,kallianpur2023weak,nasem2020social,penninkilampi2018association,poey2017social,shen2022associations}.
The Lancet Commission on Dementia Prevention estimates that reducing
social isolation could prevent 4\% of dementia cases \cite{livingston2020dementia}.
Thus, it is essential to explore how to deliver intervention at
population scale for older adults against social isolation and thereby dementia.

\textbf{The efficacy of conversational intervention has been proven, but its scalable delivery remains challenging.}
Recently, the I-CONECT clinical trial
(NCT02871921) provides rigorous experimental evidence that frequent,
semi-structured conversational interactions delivered via video can
benefit older adults with mild cognitive impairment (MCI)
\cite{dodge2024internet}. The trial is characterized by protocolized
delivery with trained interviewers and a controlled framework. However,
its key limitation, as a foundation for public health impact, is
feasibility for wide dissemination: interviewer-delivered programs are
workforce intensive and costly. Scaling requires recruiting, training,
and supervising staff, plus complex scheduling and quality assurance,
which can increase costs and reduce delivery consistency. As a result,
despite efficacy, the model is difficult to sustain for socially
isolated older adults who need consistent, high-frequency engagement.

\textbf{AI-based conversational solution is scalable but under explored for intervention and concerned with safety.}
Recent conversational AI systems can generate highly fluent
conversations at low marginal cost, suggesting a plausible path to scale conversational intervention.
For example, based on OpenAI GPT-4o pricing (July 2024
\cite{openai2024pricing}), a twice-per-week, one-year service would cost
on the order of \$7.68 in model usage fees (0.3\% of an estimated
\$2,496/year human-delivered service cost, based on 2024 salary estimates
\cite{ziprecruiter2024chat}). 
Yet the rigor of evidence for AI-based conversational intervention remains limited:
it is not established that an AI system can reliably implement the
I-CONECT interview protocol, sustain engagement in older
adults with MCI or NC, or operate safely in real-world
settings. 
Our preliminary work suggests that Large Language Models (LLMs) can be
steered via customized prompts to increase engagement in simulated
conversations with simulated participants \cite{hong2024aconect}. Though it showed a promising way to develop the chatbot, it was under-optimized for older
adults (e.g., limited modality/device adaptation, engagement and therapy implementation), and its safety and feasibility in the target population is unknown. Specifically, deployment with vulnerable older adults poses higher safety risks (e.g., misinformation,
biased or harmful content, privacy breaches, emotional distress) and therefore the feasibility could be concerned. 
Thus, rigorous development and evaluation are needed to determine whether
AI-based conversational intervention can be a scalable, safe, and
effective alternative to human-delivered programs.

\textbf{AI-CONECT aims to translate I-CONECT to scalable, safety-aware delivery and evaluate its feasibility.}
This R21 directly addresses the scalability barrier by translating the
evidence-based I-CONECT conversational protocol into an AI-delivered
system (AI-CONECT) designed for older adults. 
If successful, this project will shift the field from
workforce-limited, interviewer-dependent conversational interventions to
protocol-faithful, on-demand delivery that can be disseminated at scale.
\emph{Scientifically}, the project will improve knowledge of conversational intervention by evaluating if
AI-based conversational system is feasible and satisfiable to execute
I-CONECT intervention protocols in socially isolated older adults with MCI or normal cognition. 
\emph{Technically}, it will advance the state of AI-based conversational
interventions by directly addressing key implementation challenges:
maintaining protocol fidelity to the I-CONECT workflow, enhancing
engagement during extended sessions through cognitively demanding
prompts and turn-taking support, reducing technology barriers via a
voice-forward multimodal interface, and incorporating safety-aware
monitoring to mitigate risks (e.g., misinformation, financial
exploitation, or emotional distress) during home use.
\emph{Clinically}, it will pave ways for the use of AI in future behavioral interventions.

% If successful, this project will shift the field from
% workforce-limited, interviewer-dependent conversational interventions to
% protocol-faithful, on-demand delivery that can be disseminated at scale. It would enable larger and more
% diverse prevention studies, reduce per-participant delivery costs, and
% support future implementation of conversational engagement as a broadly
% accessible service for older adults at risk for cognitive decline.

\section{INNOVATION}

\textbf{Clinical Innovation: Translating human-delivered intervention to AI-based conversational system.}
AI-CONECT is innovative in directly translating a proven, interviewer-led dementia prevention intervention into a protocol-faithful, AI-delivered system tailored for socially isolated older adults. This systematic translation enables high-fidelity automation of the intervention, thereby opening a new paradigm for scaling evidence-based conversational intervention.

\textbf{Technology Innovation 1: Accelerating cohort-oriented development of AI chatbots via data-driven simulation.} 
Developing an AI chatbot for a cohort with specific needs (e.g., socially isolated older adults with mild cognitive impairment (MCI)) is limited by scarce data, high evaluation cost, and the risk of deploying immature designs. We introduce a data-driven simulation framework to model cohort-typical language and engagement behaviors and to stress-test candidate chatbot behaviors \emph{before} involving older adults. Unlike previous simulation methods~\cite{wang2024patient}, our method is grounded in recorded I-CONECT conversations (enabling us to capture unique MCI traits) and intervention goals (e.g., engagement and reminiscence). This approach refines and extends existing chatbot development methods by adding quantitative, intervention-relevant virtual evaluation environment that enable rapid iteration, transparent comparison across versions, and earlier identification of failure modes (e.g., low engagement or protocol drift), thereby reducing development time and risk.

\textbf{Technology Innovation 2: User-friendly multi-modality and voice-responsive interaction enhancing engagement.}
For socially isolated older adults with MCI, technology burden is a primary barrier to sustained engagement, particularly for systems that assume typing, app navigation, or complex setup.
When they faces a digital device, their attention can be easily distracted.
Using proper prompts/cues can effectively draw their attention and stimulate cognitive functions \cite{carey2019behavior}.
We will apply and extend multimodal human--computer interaction approaches by delivering a \emph{voice-forward} system (real-time speech input/output) with simple visual, emotion-responsive feedback designed for older-adult accessibility. 
To reduce interaction complexity while supporting engagement, we will deploy the chatbot on a ready-to-use tablet paired with an emotion-aware social robot that provides immediate affective cues. Compared with conventional chat interfaces, this design lowers operational demands and enables more natural conversational flow aligned with the I-CONECT interaction style.
To draw participants' attention and stimulate users' cognitive functions, we will leverage AI to generate personalized images based on the learned personal preferences in multiple sessions, as complementary prompts/cues to I-CONECT image and lanugage prompts.


% \textbf{Technology Innovation 2: User-friendly multi-modality and voice-responsive interface minimizing the technology barrier for older adults.}
% \comm{Need to think the difference to voice ChatGPT.}
% For socially isolated older adults with MCI, technology burden is a primary barrier to sustained engagement, particularly for systems that assume typing, app navigation, or complex setup. We will apply and extend multimodal human--computer interaction approaches by delivering a \emph{voice-forward} system (real-time speech input/output) with simple visual, emotion-responsive feedback designed for older-adult accessibility. As a novel instrumentation choice for this setting, we will deploy the intervention on a ready-to-use tablet paired with an emotion-aware social robot that provides immediate affective cues, reducing interaction complexity while supporting engagement. Compared with conventional chat interfaces, this design lowers operational demands and enables more natural conversational flow aligned with the I-CONECT interaction style.


\section{Approach}

\textbf{Overall Strategy.} We will develop an AI-based conversational system (AI-CONECT) to implement the I-CONECT intervention strategies and evaluate its feasibility and satisfaction among socially isolated older adults with MCI or normal cognition.

\textbf{C.1 Intervention Strategy}

The intervention protocol is modified from I-CONECT~\cite{dodge2024internet} to fit the AI-based intervention.
Each session will last for 15 minutes following the conversation flow: starting from greeting, image/topic presentation, open conversation where reminiscence and vocabulary enhancement will be executed spontaneously.
(1) \textit{Image and Language Prompts}. We adopt the same image and language prompts as I-CONECT to stimulate participants' interests and cognitive functions. The prompts includes topics were selected in I-CONECT study, that initiate the conversations with little motivations.
(2) \textit{Reminiscence with Visual interactions}. We will engage participants in reminiscence by asking
about their personal experiences related to the presented images and topics. With the AI tools, the participants will engage in generating personalized images based on the conversation and AI memory from prior sessions.
(3) \textit{Vocabulary Enhancement with Hints}. We will prompt the participants to describe the presented images in detail to enhance their vocabulary usage. AI will guide the participants to describe specific details by providing visual hints, e.g., zooming in on image parts.
% strategy addresses two main principles: (1)
% \emph{\ul{Cognitively-Demanding Conversation Strategies}}: This includes
% engaging users in novel chat experiences to stimulate their cognitive
% functions, which may enhance brain connectivity and resist
% neurodegeneration. Strategies of engaging includes providing visual
% information, novel chat themes and cognitively-demanding topics to
% stimulate users' cognitive activities. (2) \emph{\ul{User-Friendly
% Device}}: Frequent engagement essentials the accessibility of the
% service. Thus, we will design a user-friendly device that lowers the
% barrier for technologically challenged older adults to use the service
% at any time. Data privacy is also emphasized to build trust with users
% and comply with regulations like HIPAA. Overall, these strategies aim to
% provide cognitive benefits through interactive and engaging
% conversations tailored to the needs and capabilities of socially
% isolated older adults.

\textbf{C.2 Chatbot Development}

We develop AI-based chatbot to implement the intervention strategies with three components: (1) implementing intervention strategies via customizing AI, (2) virtual validation to drive AI optimization, (3) safety assurance, and (4) device development.
All development or deployments will be carried out in HIPAA-compliant servers to ensure data privacy.

\begin{figure}[ht]
  \centering
  \includegraphics[width=\textwidth]{figures/dev.png}
\end{figure}

\textbf{(1) Optimizing AI for Protocol Compliance via Learning from Verifiable Feedback.}
We will customize Large Language Models (LLMs) to implement the I-CONECT intervention strategies. We will first engineer prompts that instruct LLMs to carry out predefined tasks, including proposing chat themes and topics, and presenting topic-related images. To enhance LLM's capability in engaging users in 15-min conversations, we will improve the conversation engagement using heuristics and data from the I-CONECT trials. On the one hand, we will consult interviewers and related researchers (from I-CONECT team) to provide heuristic conversation strategies and encode the heuristics in the instructions for LLMs. On the other hand, we will \textit{supervised-finetune} the LLMs to learn the spontaneous language patterns, for example, the active use of filler words, from interviewers' speech in the I-CONECT conversation data. The conversation data will be processed from audio recordings, and we will develop speaker-aware Automatic Speech Recognition to extract high-quality text data for LLMs finetuning.
Then, we will train LLMs to learn from the \textit{feedback} on the compliance in conversations. Every time the LLM had a conversation with a user (participant), we will use rule-based method to rate the protocol compliance (engagement time, reminiscence-induction frequency, and user vocabulary). The ratings will be used to train a reward model that guides the LLMs to generate protocol-compliant conversations via Reinforcement Learning with (Human) Feedback (RLVF)\cite{ouyang2022training}.

\textbf{(2) Learning to Simulate Human Feedback from I-CONECT Data.}
The RLHF relies on conversation between the chatbot and real users to collect feedback. However, collecting feedback from real users is costly and risky, especially when the users are socially isolated older adults with MCI. To reduce the cost and risk, we will create virtual users that are LLMs finetuned on old adults' recorded conversations and then simulate conversations with the socially-isolated older adults to quantitatively evaluate the chatbot. The simulation will learn the I-CONECT participants' language traits that may drive the engagement designs. To provide a concrete simulation of older adults with MCI, we will validate that the virtual users can generate similar language symptoms that have been proven effective for MCI diagnosis \cite{hoang2023subject}. 

\textbf{(3) Safety Assurance via Real-time Monitoring.}
For safety purposes, we will examine the biased content generated by LLMs in the conversations. We will fix the problems by creating a cleaned dataset and continually fine-tuning LLMs to forget the contents \cite{liu2024rethinking}.
Yet, unknown risks may arise at deployment.
To eliminate such risks, we will create real-time safety guardrails that monitor the conversations and detect critical situations like low emotion, suicide, misinformation, or financial risks.
The guardrails will use generative rule-based methods, where an GuardAgent will generate guardrail rules to screen out unsafe generation in conversations\cite{xiang2024guardagent}.

\textbf{(4) Device and Interface Development}: We will develop an old-adult-centered
interface and device to facilitate the accessibility of our chatbot service.
We will minimize the operation complexities for older adults such that
older adults without computer-use experience can easily start a
conversation in natural language. The service will be available 24 hours
a day through voice, which is friendly for older adults with difficulty
typing. We will get feedback from survey participants on the
accessibility and improve the designs accordingly. Finally, we will
deliver an AI-empowered voice chatbot software that can be installed in
portable hardware, which implements I-CONECT conversation strategies and
attains high satisfaction from our survey participants.



% \textbf{(1) Implementing Intervention Strategies via Customizing AI}: We
% use the advanced Large Language Model (LLM) as the core of AI, which is
% able to communicate with humans in natural language. We will customize
% the state-of-the-art LLM (GPT-4o or its variants) to follow the
% intervention strategies from the I-CONECT project. We will engineer
% prompts that instruct LLMs to carry out predefined tasks, including
% proposing chat themes and topics, and presenting topic-related images.
% To enhance LLM's capability in engaging users in 15-min conversations,
% we will improve the conversation engagement using heuristics and data
% from the I-CONECT trials. On the one hand, we will consult interviewers
% and related researchers (from I-CONECT team) to provide heuristic
% conversation strategies and encode the heuristics in the instructions
% for LLMs. On the other hand, we will finetune the LLMs to learn the
% spontaneous language patterns, for example, the active use of filler
% words, from interviewers' speech in the I-CONECT conversation data. The
% conversation data will be processed from audio recordings, and we will
% develop speaker-aware Automatic Speech Recognition to extract
% high-quality text data for LLMs finetuning. 
% % To avoid LLMs memorizing or
% % being biased toward users' private information, for example, personal
% % experiences that should not be shared with a third party, we will use
% % privacy-preserving machine learning in the process\cite{xiang2024guardagent}.

% \textbf{(2) Virtual Validation to Drive AI Optimization}: To validate the implementation of the conversation strategies, we will
% validate the implementation of the conversation strategies, we will
% create virtual users that are LLMs finetuned on old adults' recorded
% conversations and then simulate conversations with the socially-isolated
% older adults to quantitatively evaluate the chatbot. The simulation will
% learn the I-CONECT participants' language traits that may drive the
% engagement designs. To provide a concrete simulation of older adults
% with MCI, we will validate that the virtual users can generate similar
% language symptoms that have been proven effective for MCI diagnosis
% \cite{hoang2023subject}. For safety purposes, we will examine the biased content
% generated by LLMs in the conversations. We will fix the problems by
% creating a cleaned dataset and continually fine-tuning LLMs to forget
% the contents \cite{liu2024rethinking}.

% \textbf{Virtual Evaluation during Development}. We will evaluate the chatbot in both virtual and human
% evaluations to validate if the chatbot is able to implement the same
% intervention strategies as the human interviewers. (1) \emph{\ul{Virtual
% Testing}}. In the early stage of development, we adopt a cost-effective
% method to evaluate the chatbot, where we simulate conversations between
% the designed chatbot and a virtual user. The virtual user is constructed
% as a digital simulation of a real old adult with or without mild
% cognitive impairment. Extensive simulation enables scalable tests to
% study how well the chatbot follows the intervention strategies. In the
% simulation, we will use GPT-4o to automatically examine if the chatbot
% will stick to the principles of intervention strategies to provide novel
% chat experiences and visual stimulations (through external visual
% tools).

% \textbf{Safety Guardrails.}

% \textbf{(3) Device Development}: We will develop an old-adult-centered
% interface device to facilitate the accessibility of our chatbot service.
% We will minimize the operation complexities for older adults such that
% older adults without computer-use experience can easily start a
% conversation in natural language. The service will be available 24 hours
% a day through voice, which is friendly for older adults with difficulty
% typing. We will get feedback from survey participants on the
% accessibility and improve the designs accordingly. Finally, we will
% deliver an AI-empowered voice chatbot software that can be installed in
% portable hardware, which implements I-CONECT conversation strategies and
% attains high satisfaction from our survey participants.



\textbf{C.3 Approach for Feasibility Study.} 

We will test the chatbot at the Memory Division of the Massachusetts Alzheimer's Disease Research Center (MADRC) and I-CONECT Foundation. 40 older adults will be recruited to complete
feasibility/satisfaction surveys on their overall experience in terms of
adherence, happiness, emotional lifting, empathy, and trustworthiness
(e.g., reporting harmful or hateful biases). The survey will also
provide insights into how to design the chatbot to engage humans and
increase social interactions.

\textbf{Participant Recruitment}. We will recruit a total of \textbf{40
socially isolated older adults aged $\geq75$ years} from the MGH Memory
Division and the I-CONECT Foundation. The cohort will consist of two
balanced groups: \textbf{20 individuals with mild cognitive impairment
(MCI)} and \textbf{20 cognitively normal (NC)} participants.

Inclusion criteria:

\begin{enumerate}
\def\labelenumi{\arabic{enumi}.}
\item
  Aged 75+
\item
  Clinical diagnosis of MCI
\item
  Identified as socially isolated by one of:

  \begin{enumerate}
  \def\labelenumii{\alph{enumii}.}
  \item
    score $\leq 12$ on the 6-item Lubben Social Network Scale (LSNS-6)
    \cite{lubben2006performance}
  \item
    engages in conversations lasting 30 minutes or longer, no more than
    twice per week, per subject self-report
  \end{enumerate}
\item
  GDS-15 (15-item Geriatric Depression Scale) at/below 9 (not severely
  depressed) \cite{yesavage1982development}.
\item
  Answering "Often" to at least one question on the Hughes et al.
  Three-Item UCLA Loneliness Scale \cite{russell1978developing}.
\item
  Ability to understand the research consent form.
\item
  Cognitive status assessed using the Telephone Interview for Cognitive
  Status (TICS) \cite{fong2009telephone} should be MCI or NC, depending on the group.
\end{enumerate}

Exclusion criteria:

\begin{enumerate}
\def\labelenumi{\arabic{enumi}.}
\item
  Diagnosed with dementia such as AD, ischemic vascular dementia, normal
  pressure hydrocephalus, or Parkinson's disease.
\item
  Schizophrenia, or other major psychiatric disorder defined by DSM-IV
  criteria \cite{wilson1994special}.
\item
  Medications: Frequent use of high doses of analgesics; Use of sedative
  medications except for those used occasionally for sleep ($\leq2$ per
  week); Use of unstable dosing of Cholinesterase inhibitors (need to be
  stable dosing for 2 months).
\end{enumerate}

\section{DATA PROCESSING AND ANALYSES}

\textbf{Aim 1}: We will use I-CONECT recorded conversations to evaluate
and optimize the chatbot design for better implementation of the
I-CONECT protocol. For the device interface development, we will
organize discussions with clinicians and caregivers.

\textbf{Method: Processing I-CONECT Data and Constructing Digital
Twins.} We will use the existing I-CONECT audiovisual corpus of older
adult interviews (with and without mild cognitive impairment, MCI) to
design and optimize the AI-CONECT chatbot. All data will be processed on
HIPAA-compliant servers with identifiers removed.

\begin{enumerate}
\def\labelenumi{\arabic{enumi}.}
\item
  \textbf{Data preprocessing:} Audio will be transcribed using a
  speaker-aware ASR optimized for older adult speech, separating
  interviewer and participant utterances. Transcripts will be aligned
  with metadata, and quality checked manually and automatically.
\item
  \textbf{Annotation and feature extraction:} We will extract
  conversational and linguistic features (e.g., word counts, lexical
  diversity, reminiscence markers) from transcripts. A subset will be
  manually annotated for reminiscence and cognitively demanding prompts.
\item
  \textbf{Digital twin construction:} We will create LLM-based "digital
  twins" of older adult participants by fine-tuning models on aggregated
  participant data (not verbatim transcripts) to simulate their language
  patterns and engagement behaviors, reflecting differences between MCI
  and cognitively normal participants.
\item
  \textbf{Simulation-based evaluation:} The digital twins will simulate
  large-scale conversations with the AI-CONECT chatbot, enabling
  low-cost testing and controlled experimentation by varying chatbot
  strategies.
\item
  \textbf{Virtual benchmarks:} We will establish quantitative
  performance benchmarks against I-CONECT targets, including:

  \begin{itemize}
  \item
    \textbf{Engagement:} Participant word ratio (participant words/total
    words).
  \item
    \textbf{Reminiscence:} Reminiscence turn ratio (autobiographical
    turns/total participant turns).
  \item
    \textbf{Vocabulary enhancement:} Lexical richness indices (e.g.,
    type--token ratio).
  \end{itemize}
\end{enumerate}

\textbf{Iterative optimization:} Simulation results will guide
successive optimization cycles, revising chatbot parameters that fail to
meet engagement or reminiscence thresholds. Only validated chatbot
configurations will advance to human testing.

Collectively, Aim 1 establishes a rigorous, data-driven foundation for
translating the I-CONECT intervention protocol into a scalable, AI-based
conversational system while minimizing risk prior to human deployment.

\textbf{Aim 2}: We will conduct a survey to test whether AI is a
feasible solution as an alternative to a human-led one.

We will recruit a total of \textbf{40 socially isolated older adults
aged $\geq75$ years}, including \textbf{20 individuals with mild cognitive
impairment (MCI)} and \textbf{20 cognitively normal (NC)} participants.
This sample size is designed to support the \textbf{primary aims of
feasibility and user satisfaction}, rather than to test intervention
efficacy.

\textbf{Feasibility (Adherence).} The primary feasibility outcome is
adherence to the intervention, defined at the participant level as
completion of \textbf{$\geq80\%$ of scheduled AI-CONECT sessions}. A total
sample size of \textbf{N=40} allows estimation of the overall adherence
proportion with adequate precision for feasibility assessment. For
example, assuming an expected adherence rate of approximately
\textbf{80\%}, N=40 yields a 95\% confidence interval with a half-width
of approximately \textbf{$\pm 12\%$}, which is sufficient to determine
whether adherence meets the predefined feasibility threshold. This level
of precision is appropriate for an R21 feasibility study and supports
clear \textbf{go/no-go decisions} for future scale-up.\\
Including \textbf{20 MCI and 20 NC participants} ensures that
feasibility can be assessed across two cognitively distinct but relevant
subpopulations. With \textbf{N=20 per group}, adherence rates can be
estimated separately for MCI and NC participants with acceptable
precision for feasibility purposes, allowing identification of
group-specific implementation challenges (e.g., differential session
completion, technology burden, or need for support) without powering the
study for between-group efficacy comparisons.

\textbf{Satisfactory Outcome: Acceptability} of the AI-CONECT
intervention will be evaluated using the Client Satisfaction
Questionnaire--8 (CSQ-8), a validated measure of user satisfaction for
health and behavioral interventions, with total scores ranging from 8 to
32 and higher scores indicating greater satisfaction.
\textbf{Acceptability will be defined a priori using explicit,
quantitative criteria}, such that the intervention will be considered
well accepted if the \textbf{mean CSQ-8 score is $\geq24$}, \textbf{at least
70\% of participants achieve CSQ-8 scores $\geq24$}, and \textbf{no more than
15\% of participants score $\leq20$}, indicating the absence of systematic
dissatisfaction. A total sample size of \textbf{N=40 (20 MCI and 20
cognitively normal participants)} is sufficient for this purpose because
it provides adequate precision to evaluate these criteria without
hypothesis testing. Specifically, if approximately 70\% of participants
meet the satisfaction threshold (CSQ-8 $\geq24$), a sample of 40 yields a
95\% confidence interval with a half-width of approximately
\textbf{$\pm 14\%$}, allowing us to distinguish between unacceptably low
satisfaction (e.g., $<55$--$60\%$) and satisfaction levels
consistent with good acceptability. Similarly, with N=40, the mean CSQ-8
score can be estimated with sufficient precision to determine whether it
exceeds the prespecified benchmark of 24, while also allowing assessment
of score variability and detection of potential ceiling or floor
effects. The allocation of \textbf{20 participants per cognitive
subgroup} further allows acceptability to be summarized separately for
MCI and cognitively normal older adults, enabling identification of
subgroup-specific usability or burden concerns, while remaining
explicitly \textbf{not powered for formal between-group comparisons}.
This precision-based justification aligns with the feasibility and
acceptability aims of the study and supports clear go/no-go decisions
for refinement and future trials, consistent with the exploratory intent
of the R21 mechanism.

\textbf{Exploratory Outcomes.} Additional measures, including cognitive
assessments, will be collected for exploratory and descriptive purposes
only. Instead, these measures will be used to evaluate feasibility of
remote assessment, completion rates, and data quality, and to generate
preliminary signals to inform the design of future trials.

In summary, a total sample size of \textbf{20 MCI and 20 NC
participants} is well-suited to the goals of this R21 by providing
sufficient precision to evaluate adherence-based feasibility and user
satisfaction, while remaining consistent with the exploratory and
developmental intent of the mechanism and the ``clinical trial not
allowed'' designation.

\section{POTENTIAL Risks AND ALTERNATIVE STRATEGIES}

The project
faces four main risks with corresponding alternative strategies:
\textbf{Risk 1} concerns the insufficient quality of I-CONECT audio data
for ASR and LLM fine-tuning, which will be mitigated by employing
advanced data augmentation and noise reduction, leveraging pre-trained
domain-specific models, and shifting emphasis to extensive prompt
engineering if necessary. \textbf{Risk 2} addresses the challenge of low
recruitment or poor adherence among older adults in the human survey
(Aim 2), which will be countered by utilizing existing collaborations
through the ADRC and AITC, simplifying the protocol (brief sessions,
provided tablet), and employing staged recruitment with early feedback.
\textbf{Risk 3} involves safety and trustworthiness concerns regarding
the LLM\textquotesingle s potential to generate harmful or non-factual
content, which will be managed through robust pre- and post-generation
content filtering, implementing limited, privacy-preserving memory
management, and integrating an in-app reporting mechanism for real-time
feedback. Finally, \textbf{Risk 4} is the difficulty in achieving
human-level conversational engagement, which will be addressed through
iterative fine-tuning guided by virtual testing metrics like "Word
Ratio" and digital twin satisfaction, post-survey algorithmic
adjustments based on human feedback on empathy and stimulation, and
potentially exploring a human-in-the-loop hybrid approach in future work
if AI-only engagement proves insufficient.

\section{IMPACT}

Our team is highly interdisciplinary, comprising a
statistician, a neurologist, and multiple computer scientists in trial
design/analytics, all with extensive experience in AD research. This R21
is especially timely, addressing the growing demand for complementary
alternatives to traditional human-led interventions. If feasible and
successful, it will greatly increase the accessibility of early dementia
intervention for socially isolated older adults.