\textbf{Specific Aims}

\textbf{Significance}. Alzheimer's disease and related dementias (ADRD)
remain a major and growing public health burden, and the disease's
multifactorial biology continues to outpace the impact of available
therapies\cite{zhang2024recent}. Current U.S. Food and Drug Administration-approved
symptomatic treatments offer modest benefits, while newer disease-modifying anti-amyloid immunotherapies are
limited to biomarker-confirmed early Alzheimer's disease and are
accompanied by restricted patient eligibility and significant treatment
burden with cost-effectiveness concerns\cite{mangalagiu2025pharmacological}. 
Thus, it is important to develop accessible and affordable prevention strategies early in the disease course.

\textbf{Critical Barriers.} Social isolation is a recognized, modifiable
risk factor for cognitive decline and dementia in older adults
\cite{evans2019social,kallianpur2023weak,nasem2020social,penninkilampi2018association,poey2017social,shen2022associations}, effective interventions that can deliver frequent social engagement at the population scale remain limited.
The recent clinical trial (I-CONECT; NCT02871921) proved that
semi-structured conversational engagement with cognitive stimulation can mitigate social isolation and cognitive decline among socially isolated older adults with Mild Cognitive Impairment (MCI)\cite{dodge2024internet}. Yet, \textit{human-delivered intervention is constrained by workforce availability and high cost}, rendering a persistent barrier to broad dissemination.
Facing the challenge, conversational AI emerges with the potential to provide scalable conversational engagement, but most existing AI systems\cite{ouyang2022training} did not encode the essential scientific methods for dementia prevention. 
In our preliminary study\cite{hong2024aconect}, we showed that Large Language Models (LLMs) can be customized to approximately follow the I-CONECT protocol in simulated textual conversations. 
However, compared to human-led conversations, the system was still under-optimized for engaging older adults including limited interactivity,  technical barriers for older adults to operate computers, unassured compliance to the I-CONECT protocols and safety rules, not validated feasibility.

\textbf{Solutions.} To enhance the accessibility and affordability of
the I-CONECT intervention, we propose to develop an Conversational AI with Multimodal Interaction (AIMI-CONECT) by leveraging advanced
Generative AI to simulate the human-delivered conversational engagement (I-CONECT). 
To address the limitations in our preliminary study\cite{hong2024aconect}, our specific aims are:

\textbf{Aim 1: Develop a conversational AI with multi-modality interaction (AIMI-CONECT) for executing the I-CONECT intervention.} 
We will use an agentic AI system driven by an LLM to follow the I-CONECT protocol.
(i) \textbf{Multi-modality interaction} augments user engagement by human-AI collaborative recreation and exploration of 3D reminiscence scene, voice chat and visual affective responses\cite{zhao2025transferring}.
(ii) \textbf{Usability}: To reduce the use barrier for older adults, we will integrate an ready-to-go social robot with real-time and responsive audio interface without typing or frequent button presses. 
(iii) \textbf{Protocol Compliance}:
We will utilize machine learning algorithms to align the LLM behaviors to protocol, and adopt real-time in-conversation guardrails to monitor critical situations including goal drifting away from intervention, negative affection and other safety risks.

\textbf{Aim 2: Conduct a feasibility study among 40
socially isolated older adults (age $\geq$ 75) with MCI or normal
cognition to evaluate the AI and identify areas for refinement.}
We will recruit participants to use our chatbot and complete brief
surveys. Each participant will receive 24 sessions in 6 weeks (4
sessions per week). Each session will be an independent 15-minute
conversation with the chatbot. A staff member will complete weekly
15-minute check-in phone calls to assess emotional status and potential
risks in using AI. We will collect both objective data and subjective
surveys to assess feasibility, acceptability, and safety. (i)
\textbf{Adherence}. We will collect the adherence rate, the proportion of
participants who complete $\geq$80\% of sessions, and an engagement
measure (participant word ratio). (ii) \textbf{Acceptability}. Each
participant will complete a survey after 6 weeks that assesses
satisfaction, willingness to use the device in the future, perceived
empathy, and preferences in interaction modality. \comm{to align with approach and data analysis.}

\textbf{Impact}. Our team is highly interdisciplinary, comprising
computer scientists, a neurologist, and a statistician in study design and
analytics, all with extensive experience in Alzheimer's disease
research. This proposed project is especially timely,
addressing the growing demand for complementary alternatives to
traditional human-led interventions. If feasible and acceptable,
AIMI-CONECT will enable scalable conversational engagement for socially
isolated older adults and inform future, large-scale studies.


\cleardoublepage
\section{Significance}

Alzheimer's disease and related dementias (ADRD) is a world-wide disease yet difficult to treat, because available pharmacologic options are limited by side effects, high cost, and restricted eligibility for newer disease-modifying therapies\cite{zhang2024recent}. 
As millions of patients remain without accessible treatment
options, it was estimated that up to $\sim 40\%$ of AD risk
may be attributable to modifiable factors. 
For example, epidemiological research has associated social isolation with greater risk of cognitive decline and dementia \cite{evans2019social,kallianpur2023weak,nasem2020social,penninkilampi2018association,poey2017social,shen2022associations}. Meanwhile, it was estimated that reducing social isolation could approximately prevent 4\% of dementia cases\cite{livingston2020dementia}, motivating the development of social isolation targeted behavioral interventions\cite{dodge2015web,yu2021iconect,otake2021picmor,watanabemiura2025picmoa}.

\textbf{Conversational intervention was proven effetive, but its scalable delivery remains challenging.}
Recently, the Internet-based conversational intervention, I-CONECT (NIH-funded clinical trial NCT02871921) which uses frequent semi-structured conversational interactions, was proven to be effective in mitigating cognitive decline among socially isolated older adults with mild cognitive impairment (MCI)\cite{dodge2024internet}.
However, its key limitation, as a foundation for public health impact, is
feasibility for wide dissemination: the intervention relies on well-trained interviewer to deliver, which is workforce intensive and costly. 
Scaling the intervention requires recruiting, training, and supervising staff, plus complex scheduling and quality assurance, which can increase costs and reduce delivery consistency and quality. 

\textbf{Conversational AI provides a scalable alternative but is under explored.}
Confronting the challenge, the emergence of conversational AI driven by Large Language Models (LLMs), that can chat fluently at low cost and replicatable with consistent quality\cite{ouyang2022training}, has fostered new opportunities for scalable conversational intervention.
The estimated cost reduction is significant: based on OpenAI GPT-4o pricing (July 2024
\cite{openai2024pricing}), a twice-per-week, one-year service would cost
0.3\% of an estimated human-delivered service cost, based on 2024 salary estimates\cite{ziprecruiter2024chat,hong2024aconect}.
Despite the promise, most conversational AI around these days do not have any scientific background nor evidence-based efficacy proven approach to address cognitive functions.
As an early attempt to applying conversational in intervention, our prior work customized LLMs to follow the I-CONECT protocol via prompt engineering\cite{hong2024aconect}, which however presents multiple limitations:
(i) The single-modality conversation has limited interactivity and thereby lowers engagement compared to human-led ones; 
(ii) Older adults faces technical barriers to operate computers as well as access the AI service, which may lead to low adherence; 
(iii) The intervention protocol was not adapted to AI, ignoring the unique challenges of AI-based intervention, for example, the potential safety risks and the protocol drift (failure to maintain protocol fidelity) during conversations\cite{zou2023universal,arike2025goaldrift}.
When older adults are socially isolated and thereby are less likely to be educated to handle such harzards, the concerns in the reliability and safety of the AI-based intervention will be further amplified.
(iv) The system has not been evaluated among the targeted population yet, so the feasibility of AI-based intervention is doubtful. 

\textbf{AIMI-CONECT aims to translate I-CONECT to scalable delivery.}
This project directly addresses the scalability barrier by translating the
evidence-based I-CONECT conversational protocol into an AI-delivered
system with multimodal interaction (AIMI-CONECT) designed for older adults. 
If successful, this project will shift the field from
workforce-limited conversational interventions to protocol-faithful, accessible and on-demand delivery that can be disseminated at scale.
\emph{Scientifically}, the project will improve knowledge of conversational intervention by evaluating if
AI-based conversational intervention is feasible in socially isolated older adults with MCI or normal cognition. 
\emph{Technically}, it will advance the state of AI-based conversational
interventions by directly addressing key challenges:
maintaining protocol compliance and usability, and enhancing engagement during extended sessions through multimodal interaction.
\emph{Clinically}, it will pave ways for the use of AI in future behavioral interventions.

\section{INNOVATION}

\textbf{From human-delivered to AI-based conversational intervention.}
AIMI-CONECT translates a efficacy-proven, interviewer-led dementia prevention intervention into a protocol-faithful, scalable AI-delivered system tailored for socially isolated older adults. This systematic translation adapts I-CONECT protocol to AI addressing emergent safety risks and evaluate its feasibility to enable high-fidelity automation of the future intervention.

\textbf{Multimodal interaction enabled conversational AI system improving usability and engagement.} 
To mitigate the gap between human and AI engagement, particularly the gap caused by the low usability for older adults, AIMI-CONECT integrates multimodal interaction into the conversational system: (1) a \emph{voice}-forward interface (real-time speech input/output); (2) visual affective robot; (3) user-AI co-recreation of reminiscence 3D scene via conversational instructions. 
Compared with conventional chat interfaces, the compound design lowers operational demands enhancing usability, and enables more engaging conversations aligned with the I-CONECT goals.
Unique to this work, the 3D scene recreation allows the participants to actively engage in the reminiscence process as well as detailed item describing with enhanced vocabulary, which are key therapeutic components of the I-CONECT protocol, and thereby may strengthen the cognitive stimulation effects.

\textbf{Enhancing protocol compliance of AI via learning from clinical trial grounded conversation simulation.}
We leverage reinforcement learning from human feedback (RLHF)\cite{ouyang2022training} to align the LLM-based chatbot to the I-CONECT protocol.
Yet, when targeting the specific population (e.g., socially isolated older adults with mild cognitive impairment (MCI)), traditional RLHF is limited by scarce data and high cost to collect human feedback.
Thus, in the project, we introduce a data-driven simulation framework that can generate conversations between LLMs and virtual users that are LLMs fine-tuned on recorded conversations from the I-CONECT clinical trial.
Unlike previous simulation methods~\cite{wang2024patient}, our method is grounded in clinical trial (enabling us to capture unique MCI traits).
This approach refines and extends existing AI development methods by adding quantitative, intervention-relevant virtual evaluation environment, and therefore enables rapid iteration and earlier identification of risks.


\section{Approach}

\textbf{Overall Strategy.} We will develop an AI-based conversational system (AIMI-CONECT) to implement the AI-adapted I-CONECT intervention protocol and evaluate its feasibility and satisfaction among socially isolated older adults with MCI or normal cognition.

\textbf{C.1 Intervention Protocol with Safety Assurance}

The intervention protocol is modified from I-CONECT~\cite{dodge2024internet} to fit the AI-based intervention.
Each session will last for 15 minutes following the conversation flow: starting from greeting, image/topic presentation, open conversation where reminiscence and vocabulary enhancement will be executed spontaneously.
(1) \textit{Image and Language Prompting of Critical Thinking}. We adopt the same image and language prompts as I-CONECT to stimulate participants' interests and cognitive functions. The prompts includes topics were selected in I-CONECT study, that initiate the conversations with little motivations.
\comm{How to prompt critical thinking?}
(2) \textit{Reminiscence with Visual interactions}. We will engage participants in reminiscence by asking
about their personal experiences related to the presented images and topics. With the AI tools, the participants will engage in generating personalized images based on the conversation and AI memory from prior sessions.
(3) \textit{Vocabulary Enhancement with Hints}. We will prompt the participants to describe the presented images in detail to enhance their vocabulary usage. AI will guide the participants to describe specific details by providing visual hints, e.g., zooming in on image parts.
% strategy addresses two main principles: (1)
% \emph{\ul{Cognitively-Demanding Conversation Strategies}}: This includes
% engaging users in novel chat experiences to stimulate their cognitive
% functions, which may enhance brain connectivity and resist
% neurodegeneration. Strategies of engaging includes providing visual
% information, novel chat themes and cognitively-demanding topics to
% stimulate users' cognitive activities. (2) \emph{\ul{User-Friendly
% Device}}: Frequent engagement essentials the accessibility of the
% service. Thus, we will design a user-friendly device that lowers the
% barrier for technologically challenged older adults to use the service
% at any time. Data privacy is also emphasized to build trust with users
% and comply with regulations like HIPAA. Overall, these strategies aim to
% provide cognitive benefits through interactive and engaging
% conversations tailored to the needs and capabilities of socially
% isolated older adults.
(4) \textit{Safety Assurance}. Distinct from human protocol, we add safety assurance to define the prohibited hevaviors that should be avoided in the conversations, for example, generating misinformation, financial advice, or harmful content. \comm{more details?}

\textbf{C.2 System Design with Multimodal Interaction.} 

\comm{Include device, voice interface, visual interaction, prompt, etc.}

\textbf{(1) Voice interface.}

\textbf{(2) 3D Memory Scene Collaborative Recreation.}

\textbf{(3) Safety Assurance via Real-time Monitoring.}
For safety purposes, we will examine the biased content generated by LLMs in the conversations. We will fix the problems by creating a cleaned dataset and continually fine-tuning LLMs to forget the contents \cite{liu2024rethinking}.
Yet, unknown risks may arise at deployment.
To eliminate such risks, we will create real-time safety guardrails that monitor the conversations and detect critical situations like low emotion, suicide, misinformation, or financial risks.
The guardrails will use generative rule-based methods, where an GuardAgent will generate guardrail rules to screen out unsafe generation in conversations\cite{xiang2024guardagent}.

\textbf{(4) Device and Interface Development}: We will develop an old-adult-centered
interface and device to facilitate the accessibility of our chatbot service.
We will minimize the operation complexities for older adults such that
older adults without computer-use experience can easily start a
conversation in natural language. The service will be available 24 hours
a day through voice, which is friendly for older adults with difficulty
typing. We will get feedback from survey participants on the
accessibility and improve the designs accordingly. Finally, we will
deliver an AI-empowered voice chatbot software that can be installed in
portable hardware, which implements I-CONECT conversation strategies and
attains high satisfaction from our survey participants.



\textbf{C.3 Optimization Algorithms for Improving AI Protocol Compliance.} 

\comm{To maintain the best responsiveness, we only train the planning model that runs on the backend. This minimizes the changes in production likes real-time API.}
\comm{We will use RLHF to optimize the chatbot to follow the protocol. The reward model will be trained on the feedback of protocol compliance, which is rated by rule-based method. However, direct rating is expensive. Instead, we build simulation system plus LLM as evaluator on the simulated conversation. The simulation fidelity is guaranteed by fine-tuing LLMs on real clinical trial data.}

\textbf{(1) Optimizing AI for Protocol Compliance via Learning from Verifiable Feedback.}
We will customize Large Language Models (LLMs) to implement the I-CONECT intervention strategies. We will first engineer prompts that instruct LLMs to carry out predefined tasks, including proposing chat themes and topics, and presenting topic-related images. To enhance LLM's capability in engaging users in 15-min conversations, we will improve the conversation engagement using heuristics and data from the I-CONECT trials. On the one hand, we will consult interviewers and related researchers (from I-CONECT team) to provide heuristic conversation strategies and encode the heuristics in the instructions for LLMs. On the other hand, we will \textit{supervised-finetune} the LLMs to learn the spontaneous language patterns, for example, the active use of filler words, from interviewers' speech in the I-CONECT conversation data. The conversation data will be processed from audio recordings, and we will develop speaker-aware Automatic Speech Recognition to extract high-quality text data for LLMs finetuning.
Then, we will train LLMs to learn from the \textit{feedback} on the compliance in conversations. Every time the LLM had a conversation with a user (participant), we will use rule-based method to rate the protocol compliance (engagement time, reminiscence-induction frequency, and user vocabulary). The ratings will be used to train a reward model that guides the LLMs to generate protocol-compliant conversations via Reinforcement Learning with (Human) Feedback (RLVF)\cite{ouyang2022training}.
\comm{Cite my ICRA paper.}

\textbf{(2) Learning to Simulate Human Feedback from I-CONECT Data.}
The RLHF relies on conversation between the chatbot and real users to collect feedback. However, collecting feedback from real users is costly and risky, especially when the users are socially isolated older adults with MCI. To reduce the cost and risk, we will create virtual users that are LLMs finetuned on old adults' recorded conversations and then simulate conversations with the socially-isolated older adults to quantitatively evaluate the chatbot. The simulation will learn the I-CONECT participants' language traits that may drive the engagement designs. To provide a concrete simulation of older adults with MCI, we will validate that the virtual users can generate similar language symptoms that have been proven effective for MCI diagnosis \cite{hoang2023subject}. 

\textbf{C.4 Approach for Feasibility Study.} 

We will test the chatbot at the Memory Division of the Massachusetts Alzheimer's Disease Research Center (MADRC) and I-CONECT Foundation. 
40 older adults will be recruited to complete
feasibility/satisfaction surveys on their overall experience in terms of
adherence, happiness, emotional lifting, empathy, and trustworthiness
(e.g., reporting harmful or hateful biases). The survey will also
provide insights into how to design the chatbot to engage humans and
increase social interactions.

Each participant will receive 15-min conversation with our chatbot for
24 sessions in 6 weeks (4 sessions per week). A staff member will do
weekly 15-min check-in phone calls to assess the emotional status and
potential risks in using AI.
In 6 weeks, we will collect both objective data and
subjective surveys to assess the feasibility and satisfaction.

To isolate the effects of social engagement from confounding factors, the study incorporates three key design elements. First, we employ chatbot persona rotation across participants to enhance interaction variability and minimize emotional bonding confounds, with personas drawn from a pool based on I-CONECT interviewer profiles. Second, our intuitive interface and natural-sounding voice capabilities enable participants lacking internet or webcam proficiency to participate in conversations without additional cognitive burden from technology learning. Third, we leverage the standardized daily themes and image stimuli from I-CONECT to support participants in organizing and expressing their ideas within conversations\cite{dodge2024internet}.

\comm{Add details about survey. }
\comm{Example: Do you change your perception about using AI. Perecentage of attrition..}

\textbf{Participant Recruitment}. We will recruit a total of \textbf{40
socially isolated older adults aged $\geq75$ years} from the MGH Memory
Division and the I-CONECT Foundation. The cohort will consist of two
balanced groups: \textbf{20 individuals with mild cognitive impairment
(MCI)} and \textbf{20 cognitively normal (NC)} participants.

Inclusion criteria:

\begin{enumerate}
\def\labelenumi{\arabic{enumi}.}
\item
  Aged 75+
\item
  Clinical diagnosis of MCI
\item
  Identified as socially isolated by one of:

  \begin{enumerate}
  \def\labelenumii{\alph{enumii}.}
  \item
    score $\leq 12$ on the 6-item Lubben Social Network Scale (LSNS-6)
    \cite{lubben2006performance}
  \item
    engages in conversations lasting 30 minutes or longer, no more than
    twice per week, per subject self-report
  \end{enumerate}
\item
  GDS-15 (15-item Geriatric Depression Scale) at/below 9 (not severely
  depressed) \cite{yesavage1982development}.
\item
  Answering "Often" to at least one question on the Hughes et al.
  Three-Item UCLA Loneliness Scale \cite{russell1978developing}.
\item
  Ability to understand the research consent form.
\item
  Cognitive status assessed using the Telephone Interview for Cognitive
  Status (TICS) \cite{fong2009telephone} should be MCI or NC, depending on the group.
\end{enumerate}

Exclusion criteria:

\begin{enumerate}
\def\labelenumi{\arabic{enumi}.}
\item
  Diagnosed with dementia such as AD, ischemic vascular dementia, normal
  pressure hydrocephalus, or Parkinson's disease.
\item
  Schizophrenia, or other major psychiatric disorder defined by DSM-IV
  criteria \cite{wilson1994special}.
\item
  Medications: Frequent use of high doses of analgesics; Use of sedative
  medications except for those used occasionally for sleep ($\leq2$ per
  week); Use of unstable dosing of Cholinesterase inhibitors (need to be
  stable dosing for 2 months).
\end{enumerate}

\section{DATA PROCESSING AND ANALYSES}

\textbf{Aim 1}: We will use I-CONECT recorded conversations to evaluate
and optimize the chatbot design for better implementation of the
I-CONECT protocol.

\textbf{Method: Processing I-CONECT Data and Constructing Digital
Twins.} We will use the existing I-CONECT audiovisual corpus of older
adult interviews (with and without mild cognitive impairment, MCI) to
design and optimize the AIMI-CONECT chatbot. All data will be processed on
HIPAA-compliant servers with identifiers removed.

\begin{enumerate}
\def\labelenumi{\arabic{enumi}.}
\item
  \textbf{Data preprocessing:} Audio will be transcribed using a
  speaker-aware ASR optimized for older adult speech, separating
  interviewer and participant utterances. Transcripts will be aligned
  with metadata, and quality checked manually and automatically.
\item
  \textbf{Annotation and feature extraction:} We will extract
  conversational and linguistic features (e.g., word counts, lexical
  diversity, reminiscence markers) from transcripts. A subset will be
  manually annotated for reminiscence and cognitively demanding prompts.
\item
  \textbf{Digital twin construction:} We will create LLM-based "digital
  twins" of older adult participants by fine-tuning models on aggregated
  participant data (not verbatim transcripts) to simulate their language
  patterns and engagement behaviors, reflecting differences between MCI
  and cognitively normal participants.
\item
  \textbf{Simulation-based evaluation:} The digital twins will simulate
  large-scale conversations with the AIMI-CONECT chatbot, enabling
  low-cost testing and controlled experimentation by varying chatbot
  strategies.
\item
  \textbf{Virtual benchmarks:} We will establish quantitative
  performance benchmarks against I-CONECT targets, including:

  \begin{itemize}
  \item
    \textbf{Engagement:} Participant word ratio (participant words/total
    words).
  \item
    \textbf{Reminiscence:} Reminiscence turn ratio (autobiographical
    turns/total participant turns).
  \item
    \textbf{Vocabulary enhancement:} Lexical richness indices (e.g.,
    type--token ratio).
  \end{itemize}
\end{enumerate}

\textbf{Iterative optimization:} Simulation results will guide
successive optimization cycles, revising chatbot parameters that fail to
meet engagement or reminiscence thresholds. Only validated chatbot
configurations will advance to human testing.

Collectively, Aim 1 establishes a rigorous, data-driven foundation for
translating the I-CONECT intervention protocol into a scalable, AI-based
conversational system while minimizing risk prior to human deployment.

\textbf{Aim 2}: We will conduct a survey to evaluate the feasibility of using AI as an alternative to a human-led one.

We will recruit a total of \textbf{40 socially isolated older adults
aged $\geq75$ years}, including \textbf{20 individuals with mild cognitive
impairment (MCI)} and \textbf{20 cognitively normal (NC)} participants.
This sample size is designed to support the \textbf{primary aims of
feasibility and user satisfaction}, rather than to test intervention
efficacy.

\textbf{(1) Adherence.} The primary feasibility outcome is
adherence to the intervention, defined at the participant level as
completion of \textbf{$\geq80\%$ of scheduled AIMI-CONECT sessions}. A total
sample size of \textbf{N=40} allows estimation of the overall adherence
proportion with adequate precision for feasibility assessment. For
example, assuming an expected adherence rate of approximately
\textbf{80\%}, N=40 yields a 95\% confidence interval with a half-width
of approximately \textbf{$\pm 12\%$}, which is sufficient to determine
whether adherence meets the predefined feasibility threshold. This level
of precision is appropriate for an R21 feasibility study and supports
clear \textbf{go/no-go decisions} for future scale-up.\\
Including \textbf{20 MCI and 20 NC participants} ensures that
feasibility can be assessed across two cognitively distinct but relevant
subpopulations. With \textbf{N=20 per group}, adherence rates can be
estimated separately for MCI and NC participants with acceptable
precision for feasibility purposes, allowing identification of
group-specific implementation challenges (e.g., differential session
completion, technology burden, or need for support) without powering the
study for between-group efficacy comparisons.

\textbf{(2) Acceptability} of the AIMI-CONECT
intervention will be evaluated using the Client Satisfaction
Questionnaire--8 (CSQ-8), a validated measure of user satisfaction for
health and behavioral interventions, with total scores ranging from 8 to
32 and higher scores indicating greater satisfaction.
\textbf{Acceptability will be defined a priori using explicit,
quantitative criteria}, such that the intervention will be considered
well accepted if the \textbf{mean CSQ-8 score is $\geq24$}, \textbf{at least
70\% of participants achieve CSQ-8 scores $\geq24$}, and \textbf{no more than
15\% of participants score $\leq20$}, indicating the absence of systematic
dissatisfaction. A total sample size of \textbf{N=40 (20 MCI and 20
cognitively normal participants)} is sufficient for this purpose because
it provides adequate precision to evaluate these criteria without
hypothesis testing. Specifically, if approximately 70\% of participants
meet the satisfaction threshold (CSQ-8 $\geq24$), a sample of 40 yields a
95\% confidence interval with a half-width of approximately
\textbf{$\pm 14\%$}, allowing us to distinguish between unacceptably low
satisfaction (e.g., $<55$--$60\%$) and satisfaction levels
consistent with good acceptability. Similarly, with N=40, the mean CSQ-8
score can be estimated with sufficient precision to determine whether it
exceeds the prespecified benchmark of 24, while also allowing assessment
of score variability and detection of potential ceiling or floor
effects. The allocation of \textbf{20 participants per cognitive
subgroup} further allows acceptability to be summarized separately for
MCI and cognitively normal older adults, enabling identification of
subgroup-specific usability or burden concerns.
This precision-based justification aligns with the feasibility and
acceptability aims of the study and supports clear go/no-go decisions
for refinement and future trials, consistent with the exploratory intent
of the R21 mechanism.

\textbf{Exploratory Outcomes.} Cognitive assessments will be collected for exploratory and descriptive purposes only. Additional survey will be collected to understand the participants' preferences for the different compoents in the system design, to inform the future design.
% Instead, these measures will be used to evaluate feasibility of
% remote assessment, completion rates, and data quality, and to generate
% preliminary signals to inform the design of future trials.

In summary, a total sample size of \textbf{20 MCI and 20 NC
participants} is well-suited to the goals of this R21 by providing
sufficient precision to evaluate adherence-based feasibility and user
satisfaction, while remaining consistent with the exploratory and
developmental intent of the mechanism and the ``clinical trial not
allowed'' designation.

\section{POTENTIAL Risks AND ALTERNATIVE STRATEGIES}

The project
faces additional risks that are not fully resolved by the above approach,
with corresponding alternative strategies:
\textbf{(1) Poor infrastructure or environment conditions.} In real-world
home conditions (e.g., intermittent connectivity, background noise, device
microphone/speaker failures, or latency that disrupts turn-taking), usability and session completion could be reduced. As
alternatives, we will implement a more efficient conversation flow that can
operate with lower bandwidth (longer pause between turns allowing more time for caching data), provide
rapid device health checks and remote troubleshooting, and maintain spare
devices for swap-out.
\textbf{(2) Device malfunctioning}. Device could be lost, damaged, or malfunctioning during the study, which could lead to missing data and reduced adherence. As alternatives, we will maintain spare devices for swap-out, provide rapid remote troubleshooting, and implement regular device health checks to identify issues early.
\textbf{(3) LLM/API server downtime or rate limiting}. Dependence on third-party API services could lead to interruptions in chatbot availability due to server downtime or rate limiting, which could affect session completion and user experience. As alternatives, we will monitor API status continuously, implement caching strategies to handle temporary outages, and maintain a contingency plan to switch to alternative services or local models if necessary.
\textbf{(4) Model and platform drift} during the project
(e.g., changes in third-party model behavior, pricing, rate limits, or
content policies) could cause regressions in protocol compliance,
engagement, or latency. As alternatives, we will pin model versions when
possible, maintain an automated regression test suite based on the virtual
benchmarks, and keep a contingency path to switch to an alternative model
or a smaller on-premise model for core dialogue behaviors if external
dependencies become unstable.
\textbf{(5) Reluctance to use an always-listening voice system} due to
privacy perceptions (e.g., concerns about recording, surveillance, or data
sharing) could limit participation even if the system is technically
HIPAA compliant. As alternatives, we will strengthen transparency and
participant control by (i) clearly communicating what is captured and for
how long, (ii) minimizing data retention by default (store transcripts and
derived features when feasible rather than raw audio), and (iii) providing a
participant-facing privacy control (e.g., an explicit ``pause''/``mute''
feature and opt-out choices for storage).


% The project
% faces four main risks with corresponding alternative strategies:
% \textbf{Risk 1} concerns the insufficient quality of I-CONECT audio data
% for ASR and LLM fine-tuning, which will be mitigated by employing
% advanced data augmentation and noise reduction, leveraging pre-trained
% domain-specific models, and shifting emphasis to extensive prompt
% engineering if necessary. 
% \textbf{Risk 2} addresses the challenge of low
% recruitment or poor adherence among older adults in the human survey
% (Aim 2), which will be countered by utilizing existing collaborations
% through the ADRC, simplifying the protocol (brief sessions,
% provided tablet), and employing staged recruitment with early feedback.
% \textbf{Risk 3} involves safety and trustworthiness concerns regarding
% the LLM\textquotesingle s potential to generate harmful or non-factual
% content, which will be managed through robust pre- and post-generation
% content filtering, implementing limited, privacy-preserving memory
% management, and integrating an in-app reporting mechanism for real-time
% feedback. Finally, \textbf{Risk 4} is the difficulty in achieving
% human-level conversational engagement, which will be addressed through
% iterative fine-tuning guided by virtual testing metrics like "Word
% Ratio" and digital twin satisfaction, post-survey algorithmic
% adjustments based on human feedback on empathy and stimulation, and
% potentially exploring a human-in-the-loop hybrid approach in future work
% if AI-only engagement proves insufficient.

\section{IMPACT}

Our team is highly interdisciplinary, comprising a
statistician, a neurologist, and multiple computer scientists in trial
design/analytics, all with extensive experience in AD research. This R21
is especially timely, addressing the growing demand for complementary
alternatives to traditional human-led interventions. If feasible and
successful, it will greatly increase the accessibility of early dementia
intervention for socially isolated older adults.